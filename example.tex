\documentclass[letterpaper,twocolumn,openany, twoside, 8pt, usenames]{cocbook}
% \documentclass[twocolumn]{book}

\usepackage[utf8]{inputenc}
\usepackage[english, russian]{babel}
\usepackage{noto}
\usepackage{fontspec}
\usepackage{titlesec}
\usepackage{lipsum}
\usepackage{hyphenat}
\usepackage{graphicx}
\usepackage{enumitem}
\usepackage[table]{xcolor}
\usepackage{tabularx}

% \usepackage{tcolorbox}

% \setmainfont[Ligatures=TeX]{Kurier}
% \newfontfamily\cyrillicfont{Kurier}[Script=Cyrillic]
\newfontfamily\headingfont[]{Oswald-Regular.ttf}
\newfontfamily\smallheadingfont[]{NotoSerif}
\setmainfont[]{NotoSerif}
\titleformat{\section}
  {\headingfont\fontsize{20}{1em}\bfseries\color{RedCthulhu}}
  {\MakeUppercase{\chaptertitlename} \thesection:}
  {1em}
  {\MakeUppercase}
% \titleformat{\section}{\normalfont\bfseries}{\thesection}{1em}{}

\titleformat{\subsection}[hang]
  {\smallheadingfont\fontsize{14}{1em}\bfseries\color{RedCthulhu}}
  {\MakeUppercase{\chaptertitlename}\ \thesubsection:}
  {1em}
  {\MakeUppercase}
\titleformat*{\subsubsection}{\normalfont\fontsize{12}{1em}\bfseries\color{RedCthulhu}}

% \titleformat*{\section}{\LARGE\headingfont}{\MakeUppercase}
% \setmainfont{Oswald-Regular.ttf}

\renewenvironment{quote}{%
   \list{}{%
     \leftmargin0.0cm   % this is the adjusting screw
     \rightmargin\leftmargin
   }
   \item\relax
}
{\endlist}

\let\oldquote\quote
\let\endoldquote\endquote
\renewenvironment{quote}[2][]
  {\if\relax\detokenize{#1}\relax
     \def\quoteauthor{~#2}%
   \fi
   \oldquote}
  {\par\nobreak\smallskip\hfill(\quoteauthor)%
   \endoldquote\addvspace{\bigskipamount}}

\hyphenchar\font=-1
\sloppy

\newcolumntype{Y}{>{\centering\arraybackslash}X}

\textwidth=6.29in
\linespread{1.5} 

\begin{document}
\scriptsize

\section*{\nohyphens{Что такое зов Ктулху?}}
\begin{quote}{---Г.Ф. Лавкрафт}
{\it Самая старая и самая сильная эмоция человечества - это страх. А старейшим и сильнейшим видом страха является страх перед неизвестностью.}
\end{quote}

\begin{quote}{Г.Ф. Лавкрафт, ---{\it Безымянный город}}
{\it То не мертво, что вечность охраняет, Смерть вместе с вечностью порою умирает.}
\end{quote}

{\it Зов Ктулху} - игра о тайнах, загадках и ужасе. Игроки берут на себя роль отважных исследователей и путешествуют в странные и опасные места, раскрывают зловещие заговоры и защищают мир от ужасов ночи. На своем пути они встретят существ, обитающих за пределами пространства и времени, само существование которых угрожает рассудку, омерзительных чудовищ и безумных культистов. Им предстоит отыскивать секреты, которые не должен знать ни один человек, погружаясь в тексты старинных фолиантов. Эти обычные люди встретят множество преград на своем пути, но их героические действия вполне могут определить судьбу мира.

Созданная Сэнди Питерсеном и впервые опубликованная в 1981 году настольная ролевая игра {\it Зов Ктулху} на протяжении более 35 лет определяла жанр и часто называется одной из лучших ролевых игр. Для тех, кто достаточно смел духом чтобы раскрыть ее тайны, эта игра готовит награды за пределами вашего воображения!

В игре, каждый игрок берет на себя роль персонажа, а один из игроков - роль рефери---Хранителя Тайного Знания (``Хранителя''), который служит модератором в игре и представляет игрокам сюжет и сеттинг в рамках которых проходит их приключение. Используя игровые кубики и несложные правила, вы определяете успех или неудачу действий персонажей, в то время как сюжет заставляет их попадать в сложные и опасные ситуации.

Этот буклет позволит вам создать персонажа для настольной ролевой игры {\it Зов Ктулху}, а также обучит основам игровых правил---достаточных для того, чтобы вы с вашими друзьями могли начать играть и насладиться стартовыми приключениями. Полноценные книги правил, а также приключения можно купить на сайте создателей игры - компании Chaosium.

\subsection*{\nohyphens{Роли игроков}}

В каждой игре игроки берут на себя одну из двух ролей: либо исследователя, либо Хранителя. Большинство берет на себя роль исследователя, потому что исследование - это основное, чем они будут заниматься: решить загадку или найти выход из сложной ситуации. Сюжеты в которых будут участвовать персонажи специально будут стараться бросить им вызов: они могут оказаться ранены, испытать неподвластные рассудку видения или даже могут быть проглочены чудовищем! По мере того, как игра будет развиваться, исследователи смогут узнать о странной магии и ужасных и чуждых созданиях, получить тайные знания из магических гримуаров, полных зловещих тайн и улучшить свои навыки, по мере того, как они становятся более опытными и готовыми к приключениям.

Один игрок берет на себя роль Хранителя. Он выбирает сценарий для игры или может самостоятельно подготовить свой собственный оригинальный сюжет. В игре, Хранитель подготавливает сцену, описывает происходящее и изображает людей, которых встречают исследователи (они обычно называются NPC, от английского ``Non-Player Character'' - персонаж, роль которого принадлежит Хранителю). Хранитель также помогает определять результат действий и служит арбитром в части игровых правил. Так как Хранитель должен тратить дополнительное время и силы на подготовку к игре, игроки часто меняют роль Хранителя от сценария к сценарию. О Хранителе можно думать как о режиссере, который ставит фильм, актеры в котором (игроки) не знают, что будет дальше по сюжету.

Сам процесс игры представляет собой развивающееся взаимодействие между игроками---в образе их персонажей, разгадывающих тайну---и Хранителем, который представляет мир в котором представлена эта тайна.

В этой игре нет никакого игрового поля. Процесс, как правило, --- простой разговор: Хранитель представляет какую-то ситуацию, а затем игроки, от лица своих персонажей, говорят о том, что они хотят сделать.

Используя систему правил, чтобы все было честно и последовательно, Хранитель сообщает им, могут ли они совершить то, что хотят и если да, что нужно сделать, чтобы добиться желаемого эффекта. Как правило, это подразумевает бросок кубиков. Кубики позволяют определить результат событий и разнообразных ситуаций, а также гарантируют непредвзятость происходящего, а также добавить драмы и накала страстей --- результат броска может означать непредвиденный сюрприз, жестокое поражение или избежание смерти буквально на волосок! После того как результат броска определен, Хранитель описывает, что происходит, спрашивая у игроков, как реагируют их персонажи и так далее.

Цель ролевой игры - повеселиться. Даже когда сердца бьются, а со лба капает пот - людям нравиться пугаться, если только ужасы не настоящие. Для некоторых, расслабление после пережитого страха --- само по себе желанный результат. Другим просто нравится пребывать в этом состоянии. {\it Зов Ктулху} сможет предоставить всем причастным и то и другое!

\subsection*{\nohyphens{Игровые концепции}}

{\it Зов Ктулху} основан на историях писателя Говарда Филлипса Лавкрафта (20 августа, 1890 - 15 марта 1937гг.), жившего в начале 20-го века и чьи истории имели в основе оригинальную мрачную философию, говорившую о том, что мироздание безжалостно и им правят неизвестные человеку чуждые и обладающие богоподобной мощью создания. Эта концепция была достаточно увлекательной чтобы увлечь поколения писателей, начавших развивать и расширять лавкрафтовскую вселенную которая получила название ``Мифология Ктулху''.

Во время своей жизни, Лавкрафт был практически неизвестен и публиковался исключительно в дешевых бульварных журналах. В итоге он умер в нищете, но в наше время он считается одним из наиболее влиятельных авторов, творивших в жанре ужасов. Несмотря на его устаревшие и во многом расистские взгляды, мы можем вдохновляться творчеством Лавкрафта и использовать его как основу для ролевых игр. Творчество Лавкрафта написано тяжелым, архаичным языком, как будто сошедшим со страниц тех древних книг, которые часто упоминаются в его рассказах. Однако, мы можем заимствовать из его творчества не только ужасы, но и юмористическую сторону: Лавкрафт зачастую подтрунивал сам над собой и над своими коллегами по перу, пародируя их стиль в своем творчестве. Современные авторы книг и игр используют не только творчество Лавкрафта, но и других его современников, исследуя как современные, так и исторические социальные проблемы.

Если вы хотите узнать больше о Лавкрафте и Мифологии Ктулху - вашим лучшим помощником станет интернет. Начать лучше с Википедии.

\subsubsection*{\nohyphens{Мифология Ктулху}}

Авторство термина ``Мифология Ктулху``, как правило, присваивают Огасту Дерлету - писателю и раннему поклоннику творчества Лавкрафта. Он стал основателем издательского дома ``Arkham House``, который поставил своей целью сделать широко доступным его книги. Сегодня под этим термином подразумевается определенная вымышленная космология, которая включает в себя богов и чудовищ, тайное знание и тематику, которую называют ``лафкрафтовскими ужасами``. В ней человечеству отведена невзрачная роль быть на задворках вселенной.

Лавкрафт писал: ``Все мои рассказы основаны на предположении, что обычные человеческие законы, интересы и эмоции не имеют никакого отношения или значения в контексте мироздания``. Он также полагал, что основополагающие истины вселенной настолько чужды и ужасающи для человеческого восприятия, что малейшее их воздействие на человека может привести к потере разума. И хотя человечество хочет одновременно и покоя и истины, в космологии Лавкрафта можно обладать только одним. Человеческий разум - несовершенное вместилилище и не может одновременно содержать космические истины и неповрежденный рассудок.

В Мифологии Ктулху, те люди, что жаждут власти могут избрать путь во тьму и полностью лишиться рассудка в обмен на овладение умением повелевать тайнами времени и пространства. После заключения сделки с дьяволом, эти беспощадные чародеи могут призвать разрушения, которых этот мир еще не видел в обмен на еще большую власть и знания. Чуждые существа Мифологии зачастую безразличны к человечеству, которое часто воспринимает их как божеств и основывает культы, почитающие их. Подобные культисты - одни из основных антагонистов в этой игре.

\subsubsection*{\nohyphens{Тайна и Исследование}}

Приключения в {\it Зове Ктулху} обычно разворачиваются вокруг тайны, а к персонажам-исследователям обращаются, чтобы добраться до правды. Зачастую загадка была порождена преступлениями безумного идолопоклонника или культа в подчинении божеств Мифологии. Задача исследователей - найти улики, каждая из их которых укажет на дополнительные ответы на стоящие перед персонажами вопросы или, вероятно, поставит перед ними новые. По мере того, как исследователи находят все большее количество улик, их понимание ситуации улучшается, пока они не найдут разгадку, которая в свою очередь может привести к изначальному виновнику и финальному противостоянию. Задача игроков, которые отыгрывают этих исследователей, - раскрыть тайну и определиться с тем, как они привлекут виновных к ответственности или расправятся с ними. Иногда тайна может быть результатом странного колдовства, действий омерзительного монстра или другого необычного события, каждое из которых может оказать непредвиденный эффект на исследователей.

Само собой, все приключения разные и не все начинаются с тайны. Иногда сценарий просто начинается сценой, в которой персонажи находятся в центре опасной ситуации из которой им надо выйти. Каждое приключение, как правило, - небольшая история и эти истории отличаются друг от друга. Истории так же могут быть связаны друг с другом, и формировать более масштабный сюжет, который называют кампанией.

\section*{\nohyphens{Создание исследователя}}

Для того, чтобы играть в {\it Зов Ктулху}, вам необходимо создать персонажа - исследователя. Создание персонажа можно представить в виде формальных шагов, они описаны ниже. Игроки записывают информацию о своих персонажах на специальных листах персонажей. Эти листы содержат всю информацию, необходимую для игры. Их можно скачать здесь: https://www.chaosium.com/cthulhu-character-sheets.
Представленный ниже процесс создания персонажа немного упрощен, чтобы дать вам возможность начать играть как можно быстрее. Более подробный процесс описан в {\it Книге Хранителя} и {\it Книге Исследователя}.

\begin{figure}
  \includegraphics[width=\linewidth]{img/cthulhu.jpg}
  \caption{A boat.}
  \label{fig:boat1}
\end{figure}

% \begin{lstlisting}
\begin{figure}
\setthemecolor[CoCPaperBox]

% \begin{paperbox}{This Sidebar Is Also Mauve}
%   The sidebar is also using the new theme color.
% \end{paperbox}
\end{figure}
% \end{lstlisting}

\subsection*{Шаг первый: характеристики исследователя}
Итак, исследователь в Зове Ктулху имеет восемь характеристик:

\begin{itemize}[leftmargin=4mm]
  \item Сила (Strength, STR, СИЛ) служит мерилом чистой физической мощи вашего исследователя
  \item Телосложение (Constitution, CON, ТЛС) определяет стойкость персонажа и его здоровье
  \item Сила Воли (Power, POW, ВОЛ) - сочетание силы духа, воли и ментальной стабильности
  \item Ловкость (Dexterity, DEX, ЛВК) - мерило физической ловкости и скорости
  \item Внешность (Appearance, APP) отвечает за физическую привлекательность вашего персонажа
  \item Размер (Size, SIZ, РЗМ) отражает сочетание высоты и веса исследователя
  \item Интеллект (Intelligence, INT, ИНТ) - приблизительная прикидка хитроумия вашего персонажа и его способности делать логические и интуитивные выводы
  \item Образование (Education, EDU, ОБР) - аккумулированный жизненный опыт персонажа, будь то результат образования или жизни на улице.
\end{itemize}

Распределите следующие значения как вам угодно среди восьми характеристик: 40, 50, 50, 50, 60, 60, 70, 80. Каждое из этих значение - это процент, т.е. если вы определяете силу персонажа в 70, это означает, что у них ``СИЛ 70\%`` - показатель выше среднего, т.е. персонаж довольно силен. Запишите эти значения в большой ячейке рядом с каждой из характеристик. Эти значения называются ``Обычными''

\subsubsection*{\nohyphens{Половинные и Пятые Значения Характеристик}}

Взглянув на лист исследователя, вы увидите три ячейки рядом с каждой характеристикой: одну большую ячейку (в которой вы только что записали числа для этой характеристики) и еще две маленькие ячейки сбоку. Верхняя маленькая ячейка для ``Половинного'' значения, та что ниже - для ``Пятого''.

\begin{itemize}[leftmargin=4mm]
  \item Для того, чтобы использовать Половинное значение характеристики, просто разделите пополам значение этой характеристики, округляя в меньшую сторону до ближайшего целого значения, если необходимо. Например, если вы выбрали значение СИЛ 70, то Половинное значение (используемое для Сложных бросков) будет равно 35.
  \item Для получения Пятого значения, разделите Обычное значение на 5, также округляя в меньшую сторону. Например, для СИЛ 70 Пятое значение (используемое для Экстремально сложных бросков) будет равно 14.
\end{itemize}

Запишите Половинные и Пятые значения для каждой характеристики в лист исследователя. На стр 23 есть таблица со значениями, позволяющими быстро сориентироваться. Загружаемые PDF листов делают автоматически калькулируют требуемые значения.

\subsection*{Шаг второй: вторичные атрибуты}

Также есть определенное число атрибутов, определяемых после того, как вы определились с характеристиками вашего исследователя. Это бонус к урону (Damage Bonus), комплекция (Build), очки здоровья (Hit Points), скорость передвижения (Move Rate), рассудок (Sanity), очки магии (Magic Points). Кроме этого, вам нужно будет определить показатель удачи.
\smallbreak
\noindent \textbf{Бонус к урону (БУ) и Комплекция:} бонус к урону определяет то, как много дополнительного урона вам исследователь наносит, когда он совершает успешную атаку в ближнем или рукопашном бою. Комплекция - это результат сочетания размера и силы, который обычно используется в ``рукопашных маневрах'' (см. стр. 19). Сложите ваши Обычные значения СИЛ и РЗМ и обратитесь к таблице на стр. 8. На листе исследователя есть ячейки для итогового значения Комплекции и бонуса к урону.
\smallbreak
\noindent \textbf{Очки здоровья (ОЗ):} они являются результатом суммы РЗМ и ТЛС, разделенной на 10 и округленной вниз до ближайшего целого числа. По мере того, как ваш исследователь получает урон в бою или от каких-либо других воздействий, это значение будет уменьшаться.
\smallbreak
\noindent \example{РЗМ 50 и ТЛС 50 в сумме дают 100, разделенное на 10, это значение равно 10 очкам здоровья. Исследователь может получить 10 очков урона, после чего он потеряет сознание и, вполне вероятно, умрет.}
\smallbreak
\noindent \textbf{Скорость перемещения (ПРМ):} все персонажи-люди перемещаются со Скоростью, равной 8.
\smallbreak
\noindent \textbf{Очки рассудка (РАС):} изначально равно показателю ВОЛ персонажа. Вы увидите раздел с очками рассудка на листе персонажа: вам нужно обвести кружком нужное значение. Это значение используется в процентном (1D100) броске, который используется чтобы понять насколько ваш персонаж может сохранять хладнокровие перед лицом абсолютного кошмара. По мере того, как вы сталкиваетесь с ужасами Мифологии Ктулху, ваш показатель рассудка будет изменяться.
\smallbreak
\noindent \example{ВОЛ 40 дает показатель рассудка равный 40. Когда вы делаете бросок на Рассудок, вам нужно выкинуть меньше или равно 40 на 1D100 чтобы бросок считался успешным. Значение 41 или выше означает, что бросок на рассудок провалился.}
\smallbreak
\begin{cocpaperbox}{}{}
\raisebox{-10mm}{\includegraphics[width=\linewidth]{img/top.png}}
\cocpaperboxtitle{Таблица бонуса к урону и Комплекции}
\noindent \example{Лёша выбирает СИЛ равную 60 и РЗМ равный 70, в сумме дающие 130. Когда он совершает успешную физическую атаку, он наносит дополнительные 1D4 очка урона (Бонус к Урону). Его комплекция равна 1.}
\bigbreak
\rowcolors{2}{}{TableRow}
\begin{tabularx}{74mm}{@{}YYY@{}}
\rowcolor{TableHeading}\color{CoCPaperBox}\bfseries СИЛ + РЗМ & \color{CoCPaperBox}\bfseries Бонус к Урону & \color{CoCPaperBox}\bfseries Комплекция \\
2-64 & -2 & -2 \\
65-84 & -1 & -1 \\
85-124 & Нет & 0 \\
125-164 & +1D4 & 1 \\
165-204 & +1D6 & 2 \\
\end{tabularx}
\includegraphics[width=\linewidth]{img/bottom.png}
\end{cocpaperbox}
\noindent \textbf{Очки магии (ОМ):} равны 1/5 ВОЛ. Обведите в кружок это значение на листе исследователя. Очки магии используются в колдовстве, а также для подпитки магических устройств и волшебных эффектов. Потраченные очки восстанавливаются сами со скоростью 1 очко в час. Если у персонажа не осталось очков магии, все последующие очки вычитаются из его очков здоровья, любая подобная потеря отражается физическим уроном в том виде, в котором реши Хранитель.
\smallbreak
\noindent \example{ВОЛ 40 предоставляет 8 очков магии. При колдовстве заклятия необходимо потратить 2 очка. Количество имеющихся очков у исследователя временно падает до 6.}
\smallbreak
\noindent \textbf{Показатель удачи (Удача):} определите его, кинув 3D6 и умножьте результат на 5. Обведите итоговое значение на листе исследователя. Бросок на показатель Удачи обычно используется, чтобы определить, насколько расположены в вашу или не вашу пользу внешние обстоятельства. Как и с Рассудком, вы должны выкинуть меньше или равно показателю Удачи, чтобы бросок считался успешным. На стр. 15 приведены подробности о броске на Удачу.
\smallbreak
\noindent \example{Персонаж Романа бежит от толпы зомби и запрыгивает в ближайший автомобиль. Хранитель просит Рому сделать бросок Удачи, чтобы определить, находятся ли ключи в зажигании (т.к. это абсолютно случайная вероятность). Роман делает процентный бросок, кидая 1D100 и выкидывает 28, что меньше его показателя Удачи --- он поворачивает ключ и двигатель заводится!}

\subsection*{Шаг третий: род деятельности и навыки}

К этому моменту у вас уже должно быть представление, чем вам исследователь может зарабатывать себе на жизнь. Помните, что ``исследователь'' не обязательно означает палеонтолога или журналиста. Выбор вида занятости определит навыки, доступные вашему персонажу. Для начала нужно выбрать род деятельности. Все, кого вам было бы интересно отыгрывать, подойдет, но итоговый выбор нужно согласовать с Хранителем. Наиболее популярными видами деятельности в {\it Зове Ктулху} являются Профессор, Журналист, Оккультист и Археолог, но в целом ваш выбор ограничен только фантазией.

Либо выберите вашу профессию из приведенного списка и используйте ее навыки, либо, используя общий список навыков создайте свою собственную. Для этого выберите восемь навыков, которые наиболее подходят для вашей профессии. Эти навыки будут ``профессиональными навыками''.
\smallbreak
\noindent \textbf{Примечание:} краткое описание разных навыков может быть найдено на стр. 10-12.
\smallbreak

Теперь вы можете распределить очки навыков на листе персонажа. Ни один персонаж не может добавлять очки к навыку Знания Мифологии Ктулху, т.к. подразумевается, что любой персонаж ничего не знает об ужасах мироздания.

Распределите следующие значения по восьми профессиональным навыкам, а также по навыку Благосостояния: на один из навыков 70\%, на два 60\%, на три 50\% и еще на три 40\% (установите значения выбранных навыков равные именно этим значениям, не обращайте внимания на базовые значения навыков, указанные на листе персонажа).

\smallbreak

\noindent \example{Полина хочет играть за Журналистку и распределяет следующие навыки: Искусство/Ремесло (Фотография) 50\%, История 40\%, Использование Библиотек 50\%, Родной Язык (Английский) 60\%, Психология 40\%. Для общения она выбирает Убеждение 70\% (очень убедительная!). Затем она смотрит на лист персонажа и видит еще два навыка, которые могут оказаться полезными журналисту: Наблюдательность 50\% и Скрытность 60\%. У нее осталось одно значение - 40\%, которое она решает назначить своему Благосостоянию. Полина записывает эти значения в большую ячейку рядом с каждым навыком.}

\smallbreak

После распределения очков по профессиональным навыкам, выберите ``личные навыки''. Это те навыки, которыми ваш персонаж овладел вне своей основной рабочей деятельности. Выберите четыре не профессиональных навыка и улучшите их на 20\% (прибавьте 20\% к значению, указанному для выбранного навыка на листе персонажа).

Мы рекомендуем записывать значения ваших навыков в том же формате, что и Характеристики -  Обычное / Половинное / Пятое - они понадобятся вам во время игры (на стр 23 приведена Памятка со значениями). Само собой, если вам удобнее, можете высчитать эти значения, когда они вам понадобятся во время игры.

\smallbreak

\noindent \example{Андрей выбирает в качестве рода деятельности профессию Солдата. Так как этой профессии нет в этой книге, он сам решает, что ему подойдут следующие навыки: Скалолазание, Уклонение, Схватка (Рукопашный Бой), Огнестрельное Оружие (Винтовка/Дробовик), Первая Помощь, Иностранный Язык, Скрытность и Выживание. Андрей распределяет свои очки следующим образом: Скалолазание 60\%, Благосостояние 40\%, Уклонение 60\%, Схватка (Рукопашный Бой) 70\%, Огнестрельное Оружие (Винтовка/Дробовик) 50\%, Первая Помощь 40\%, Иностранный Язык 50\% (он выбирает Испанский в качестве второго языка), Скрытность 50\%, Выживание 40\%.

Затем Андрей выбирает четыре личных навыка (прибавляя 20\% к их стартовому значению на листе персонажа): Вождение Автомобиля 40\%, Механика 30\%, Наблюдательность 45\% и Прыжок 40\%. Каждое значение навыка он записывает в ячейку рядом с соответствующим названием и заодно вносит значения для Обычных/Половинных/Пятых, например ``Наблюдательность 45 (22/9).''
}

\end{document}
