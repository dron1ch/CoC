\documentclass[letterpaper,twocolumn,openany, twoside, 11pt, usenames]{cocbook}
% \documentclass[twocolumn]{book}

\usepackage[utf8]{inputenc}
\usepackage[english, russian]{babel}
\usepackage{noto}
\usepackage{fontspec}
\usepackage{titlesec}
\usepackage{lipsum}
\usepackage{hyphenat}
\usepackage{graphicx}
\usepackage{enumitem}
\usepackage[table]{xcolor}
\usepackage{tabularx}
\usepackage{multicol}

% \usepackage{tcolorbox}

% \setmainfont[Ligatures=TeX]{Kurier}
% \newfontfamily\cyrillicfont{Kurier}[Script=Cyrillic]
\newfontfamily\headingfont[]{Oswald-Regular.ttf}
\newfontfamily\smallheadingfont[]{NotoSerif}
\setmainfont[]{NotoSerif}
\titleformat{\section}
  {\headingfont\fontsize{20}{1em}\bfseries\color{RedCthulhu}}
  {\MakeUppercase{\chaptertitlename} \thesection:}
  {1em}
  {\MakeUppercase}
% \titleformat{\section}{\normalfont\bfseries}{\thesection}{1em}{}

\titleformat{\subsection}[hang]
  {\smallheadingfont\fontsize{14}{1em}\bfseries\color{RedCthulhu}}
  {\MakeUppercase{\chaptertitlename}\ \thesubsection:}
  {1em}
  {\MakeUppercase}
\titleformat*{\subsubsection}{\normalfont\fontsize{12}{1em}\bfseries\color{RedCthulhu}}

% \titleformat*{\section}{\LARGE\headingfont}{\MakeUppercase}
% \setmainfont{Oswald-Regular.ttf}

\renewenvironment{quote}{%
   \list{}{%
     \leftmargin0.0cm   % this is the adjusting screw
     \rightmargin\leftmargin
   }
   \item\relax
}
{\endlist}

\let\oldquote\quote
\let\endoldquote\endquote
\renewenvironment{quote}[2][]
  {\if\relax\detokenize{#1}\relax
     \def\quoteauthor{~#2}%
   \fi
   \oldquote}
  {\par\nobreak\smallskip\hfill(\quoteauthor)%
   \endoldquote\addvspace{\bigskipamount}}

\hyphenchar\font=-1
\sloppy

\newcolumntype{Y}{>{\centering\arraybackslash}X}

\textwidth=6.29in
\linespread{1.5} 

\begin{document}
\scriptsize

\section*{\nohyphens{Что такое зов Ктулху?}}
\begin{quote}{---Г.Ф. Лавкрафт}
{\it Самая старая и самая сильная эмоция человечества - это страх. А старейшим и сильнейшим видом страха является страх перед неизвестностью.}
\end{quote}

\begin{quote}{Г.Ф. Лавкрафт, ---{\it Безымянный город}}
{\it То не мертво, что вечность охраняет, Смерть вместе с вечностью порою умирает.}
\end{quote}

{\it Зов Ктулху} - игра о тайнах, загадках и ужасе. Игроки берут на себя роль отважных исследователей и путешествуют в странные и опасные места, раскрывают зловещие заговоры и защищают мир от ужасов ночи. На своем пути они встретят существ, обитающих за пределами пространства и времени, само существование которых угрожает рассудку, омерзительных чудовищ и безумных культистов. Им предстоит отыскивать секреты, которые не должен знать ни один человек, погружаясь в тексты старинных фолиантов. Эти обычные люди встретят множество преград на своем пути, но их героические действия вполне могут определить судьбу мира.

Созданная Сэнди Питерсеном и впервые опубликованная в 1981 году настольная ролевая игра {\it Зов Ктулху} на протяжении более 35 лет определяла жанр и часто называется одной из лучших ролевых игр. Для тех, кто достаточно смел духом чтобы раскрыть ее тайны, эта игра готовит награды за пределами вашего воображения!

В игре, каждый игрок берет на себя роль персонажа, а один из игроков - роль рефери---Хранителя Тайного Знания (``Хранителя''), который служит модератором в игре и представляет игрокам сюжет и сеттинг в рамках которых проходит их приключение. Используя игровые кубики и несложные правила, вы определяете успех или неудачу действий персонажей, в то время как сюжет заставляет их попадать в сложные и опасные ситуации.

Этот буклет позволит вам создать персонажа для настольной ролевой игры {\it Зов Ктулху}, а также обучит основам игровых правил---достаточных для того, чтобы вы с вашими друзьями могли начать играть и насладиться стартовыми приключениями. Полноценные книги правил, а также приключения можно купить на сайте создателей игры - компании Chaosium.

\subsection*{\nohyphens{Роли игроков}}

В каждой игре игроки берут на себя одну из двух ролей: либо исследователя, либо Хранителя. Большинство берет на себя роль исследователя, потому что исследование - это основное, чем они будут заниматься: решить загадку или найти выход из сложной ситуации. Сюжеты в которых будут участвовать персонажи специально будут стараться бросить им вызов: они могут оказаться ранены, испытать неподвластные рассудку видения или даже могут быть проглочены чудовищем! По мере того, как игра будет развиваться, исследователи смогут узнать о странной магии и ужасных и чуждых созданиях, получить тайные знания из магических гримуаров, полных зловещих тайн и улучшить свои навыки, по мере того, как они становятся более опытными и готовыми к приключениям.

Один игрок берет на себя роль Хранителя. Он выбирает сценарий для игры или может самостоятельно подготовить свой собственный оригинальный сюжет. В игре, Хранитель подготавливает сцену, описывает происходящее и изображает людей, которых встречают исследователи (они обычно называются NPC, от английского ``Non-Player Character'' - персонаж, роль которого принадлежит Хранителю). Хранитель также помогает определять результат действий и служит арбитром в части игровых правил. Так как Хранитель должен тратить дополнительное время и силы на подготовку к игре, игроки часто меняют роль Хранителя от сценария к сценарию. О Хранителе можно думать как о режиссере, который ставит фильм, актеры в котором (игроки) не знают, что будет дальше по сюжету.

Сам процесс игры представляет собой развивающееся взаимодействие между игроками---в образе их персонажей, разгадывающих тайну---и Хранителем, который представляет мир в котором представлена эта тайна.

В этой игре нет никакого игрового поля. Процесс, как правило, --- простой разговор: Хранитель представляет какую-то ситуацию, а затем игроки, от лица своих персонажей, говорят о том, что они хотят сделать.

Используя систему правил, чтобы все было честно и последовательно, Хранитель сообщает им, могут ли они совершить то, что хотят и если да, что нужно сделать, чтобы добиться желаемого эффекта. Как правило, это подразумевает бросок кубиков. Кубики позволяют определить результат событий и разнообразных ситуаций, а также гарантируют непредвзятость происходящего, а также добавить драмы и накала страстей --- результат броска может означать непредвиденный сюрприз, жестокое поражение или избежание смерти буквально на волосок! После того как результат броска определен, Хранитель описывает, что происходит, спрашивая у игроков, как реагируют их персонажи и так далее.

Цель ролевой игры - повеселиться. Даже когда сердца бьются, а со лба капает пот - людям нравиться пугаться, если только ужасы не настоящие. Для некоторых, расслабление после пережитого страха --- само по себе желанный результат. Другим просто нравится пребывать в этом состоянии. {\it Зов Ктулху} сможет предоставить всем причастным и то и другое!

\subsection*{\nohyphens{Игровые концепции}}

{\it Зов Ктулху} основан на историях писателя Говарда Филлипса Лавкрафта (20 августа, 1890 - 15 марта 1937гг.), жившего в начале 20-го века и чьи истории имели в основе оригинальную мрачную философию, говорившую о том, что мироздание безжалостно и им правят неизвестные человеку чуждые и обладающие богоподобной мощью создания. Эта концепция была достаточно увлекательной чтобы увлечь поколения писателей, начавших развивать и расширять лавкрафтовскую вселенную которая получила название ``Мифология Ктулху''.

Во время своей жизни, Лавкрафт был практически неизвестен и публиковался исключительно в дешевых бульварных журналах. В итоге он умер в нищете, но в наше время он считается одним из наиболее влиятельных авторов, творивших в жанре ужасов. Несмотря на его устаревшие и во многом расистские взгляды, мы можем вдохновляться творчеством Лавкрафта и использовать его как основу для ролевых игр. Творчество Лавкрафта написано тяжелым, архаичным языком, как будто сошедшим со страниц тех древних книг, которые часто упоминаются в его рассказах. Однако, мы можем заимствовать из его творчества не только ужасы, но и юмористическую сторону: Лавкрафт зачастую подтрунивал сам над собой и над своими коллегами по перу, пародируя их стиль в своем творчестве. Современные авторы книг и игр используют не только творчество Лавкрафта, но и других его современников, исследуя как современные, так и исторические социальные проблемы.

Если вы хотите узнать больше о Лавкрафте и Мифологии Ктулху - вашим лучшим помощником станет интернет. Начать лучше с Википедии.

\subsubsection*{\nohyphens{Мифология Ктулху}}

Авторство термина ``Мифология Ктулху``, как правило, присваивают Огасту Дерлету - писателю и раннему поклоннику творчества Лавкрафта. Он стал основателем издательского дома ``Arkham House``, который поставил своей целью сделать широко доступным его книги. Сегодня под этим термином подразумевается определенная вымышленная космология, которая включает в себя богов и чудовищ, тайное знание и тематику, которую называют ``лафкрафтовскими ужасами``. В ней человечеству отведена невзрачная роль быть на задворках вселенной.

Лавкрафт писал: ``Все мои рассказы основаны на предположении, что обычные человеческие законы, интересы и эмоции не имеют никакого отношения или значения в контексте мироздания``. Он также полагал, что основополагающие истины вселенной настолько чужды и ужасающи для человеческого восприятия, что малейшее их воздействие на человека может привести к потере разума. И хотя человечество хочет одновременно и покоя и истины, в космологии Лавкрафта можно обладать только одним. Человеческий разум - несовершенное вместилилище и не может одновременно содержать космические истины и неповрежденный рассудок.

В Мифологии Ктулху, те люди, что жаждут власти могут избрать путь во тьму и полностью лишиться рассудка в обмен на овладение умением повелевать тайнами времени и пространства. После заключения сделки с дьяволом, эти беспощадные чародеи могут призвать разрушения, которых этот мир еще не видел в обмен на еще большую власть и знания. Чуждые существа Мифологии зачастую безразличны к человечеству, которое часто воспринимает их как божеств и основывает культы, почитающие их. Подобные культисты - одни из основных антагонистов в этой игре.

\subsubsection*{\nohyphens{Тайна и Исследование}}

Приключения в {\it Зове Ктулху} обычно разворачиваются вокруг тайны, а к персонажам-исследователям обращаются, чтобы добраться до правды. Зачастую загадка была порождена преступлениями безумного идолопоклонника или культа в подчинении божеств Мифологии. Задача исследователей - найти улики, каждая из их которых укажет на дополнительные ответы на стоящие перед персонажами вопросы или, вероятно, поставит перед ними новые. По мере того, как исследователи находят все большее количество улик, их понимание ситуации улучшается, пока они не найдут разгадку, которая в свою очередь может привести к изначальному виновнику и финальному противостоянию. Задача игроков, которые отыгрывают этих исследователей, - раскрыть тайну и определиться с тем, как они привлекут виновных к ответственности или расправятся с ними. Иногда тайна может быть результатом странного колдовства, действий омерзительного монстра или другого необычного события, каждое из которых может оказать непредвиденный эффект на исследователей.

Само собой, все приключения разные и не все начинаются с тайны. Иногда сценарий просто начинается сценой, в которой персонажи находятся в центре опасной ситуации из которой им надо выйти. Каждое приключение, как правило, - небольшая история и эти истории отличаются друг от друга. Истории так же могут быть связаны друг с другом, и формировать более масштабный сюжет, который называют кампанией.

\section*{\nohyphens{Создание исследователя}}

Для того, чтобы играть в {\it Зов Ктулху}, вам необходимо создать персонажа - исследователя. Создание персонажа можно представить в виде формальных шагов, они описаны ниже. Игроки записывают информацию о своих персонажах на специальных листах персонажей. Эти листы содержат всю информацию, необходимую для игры. Их можно скачать здесь: https://www.chaosium.com/cthulhu-character-sheets.
Представленный ниже процесс создания персонажа немного упрощен, чтобы дать вам возможность начать играть как можно быстрее. Более подробный процесс описан в {\it Книге Хранителя} и {\it Книге Исследователя}.

\begin{figure}
  \includegraphics[width=\linewidth]{img/cthulhu.jpg}
  \caption{A boat.}
  \label{fig:boat1}
\end{figure}

% \begin{lstlisting}
\begin{figure}
\setthemecolor[CoCPaperBox]

% \begin{paperbox}{This Sidebar Is Also Mauve}
%   The sidebar is also using the new theme color.
% \end{paperbox}
\end{figure}
% \end{lstlisting}

\subsection*{Шаг первый: характеристики исследователя}
Итак, исследователь в Зове Ктулху имеет восемь характеристик:

\begin{itemize}[leftmargin=4mm]
  \item Сила (Strength, STR, СИЛ) служит мерилом чистой физической мощи вашего исследователя
  \item Телосложение (Constitution, CON, ТЛС) определяет стойкость персонажа и его здоровье
  \item Сила Воли (Power, POW, ВОЛ) - сочетание силы духа, воли и ментальной стабильности
  \item Ловкость (Dexterity, DEX, ЛВК) - мерило физической ловкости и скорости
  \item Внешность (Appearance, APP) отвечает за физическую привлекательность вашего персонажа
  \item Размер (Size, SIZ, РЗМ) отражает сочетание высоты и веса исследователя
  \item Интеллект (Intelligence, INT, ИНТ) - приблизительная прикидка хитроумия вашего персонажа и его способности делать логические и интуитивные выводы
  \item Образование (Education, EDU, ОБР) - аккумулированный жизненный опыт персонажа, будь то результат образования или жизни на улице.
\end{itemize}

Распределите следующие значения как вам угодно среди восьми характеристик: 40, 50, 50, 50, 60, 60, 70, 80. Каждое из этих значение - это процент, т.е. если вы определяете силу персонажа в 70, это означает, что у них ``СИЛ 70\%`` - показатель выше среднего, т.е. персонаж довольно силен. Запишите эти значения в большой ячейке рядом с каждой из характеристик. Эти значения называются ``Обычными''

\subsubsection*{\nohyphens{Половинные и Пятые Значения Характеристик}}

Взглянув на лист исследователя, вы увидите три ячейки рядом с каждой характеристикой: одну большую ячейку (в которой вы только что записали числа для этой характеристики) и еще две маленькие ячейки сбоку. Верхняя маленькая ячейка для ``Половинного'' значения, та что ниже - для ``Пятого''.

\begin{itemize}[leftmargin=4mm]
  \item Для того, чтобы использовать Половинное значение характеристики, просто разделите пополам значение этой характеристики, округляя в меньшую сторону до ближайшего целого значения, если необходимо. Например, если вы выбрали значение СИЛ 70, то Половинное значение (используемое для Сложных бросков) будет равно 35.
  \item Для получения Пятого значения, разделите Обычное значение на 5, также округляя в меньшую сторону. Например, для СИЛ 70 Пятое значение (используемое для Экстремально сложных бросков) будет равно 14.
\end{itemize}

Запишите Половинные и Пятые значения для каждой характеристики в лист исследователя. На стр 23 есть таблица со значениями, позволяющими быстро сориентироваться. Загружаемые PDF листов делают автоматически калькулируют требуемые значения.

\subsection*{Шаг второй: вторичные атрибуты}

Также есть определенное число атрибутов, определяемых после того, как вы определились с характеристиками вашего исследователя. Это бонус к урону (Damage Bonus), комплекция (Build), очки здоровья (Hit Points), скорость передвижения (Move Rate), рассудок (Sanity), очки магии (Magic Points). Кроме этого, вам нужно будет определить показатель удачи.
\smallbreak
\noindent \textbf{Бонус к урону (БУ) и Комплекция:} бонус к урону определяет то, как много дополнительного урона вам исследователь наносит, когда он совершает успешную атаку в ближнем или рукопашном бою. Комплекция - это результат сочетания размера и силы, который обычно используется в ``рукопашных маневрах'' (см. стр. 19). Сложите ваши Обычные значения СИЛ и РЗМ и обратитесь к таблице на стр. 8. На листе исследователя есть ячейки для итогового значения Комплекции и бонуса к урону.
\smallbreak
\noindent \textbf{Очки здоровья (ОЗ):} они являются результатом суммы РЗМ и ТЛС, разделенной на 10 и округленной вниз до ближайшего целого числа. По мере того, как ваш исследователь получает урон в бою или от каких-либо других воздействий, это значение будет уменьшаться.
\smallbreak
\noindent \example{РЗМ 50 и ТЛС 50 в сумме дают 100, разделенное на 10, это значение равно 10 очкам здоровья. Исследователь может получить 10 очков урона, после чего он потеряет сознание и, вполне вероятно, умрет.}
\smallbreak
\noindent \textbf{Скорость перемещения (ПРМ):} все персонажи-люди перемещаются со Скоростью, равной 8.
\smallbreak
\noindent \textbf{Очки рассудка (РАС):} изначально равно показателю ВОЛ персонажа. Вы увидите раздел с очками рассудка на листе персонажа: вам нужно обвести кружком нужное значение. Это значение используется в процентном (1D100) броске, который используется чтобы понять насколько ваш персонаж может сохранять хладнокровие перед лицом абсолютного кошмара. По мере того, как вы сталкиваетесь с ужасами Мифологии Ктулху, ваш показатель рассудка будет изменяться.
\smallbreak
\noindent \example{ВОЛ 40 дает показатель рассудка равный 40. Когда вы делаете бросок на Рассудок, вам нужно выкинуть меньше или равно 40 на 1D100 чтобы бросок считался успешным. Значение 41 или выше означает, что бросок на рассудок провалился.}
\smallbreak
\begin{cocpaperbox}{}{}
  \raisebox{-10mm}{\includegraphics[width=\linewidth]{img/top.png}}
  \cocpaperboxtitle{Таблица бонуса к урону и Комплекции}
  \noindent \example{Рома выбирает СИЛ равную 60 и РЗМ равный 70, в сумме дающие 130. Когда он совершает успешную физическую атаку, он наносит дополнительные 1D4 очка урона (Бонус к Урону). Его комплекция равна 1.}
  \bigbreak
  \rowcolors{2}{}{TableRow}
  \begin{tabularx}{74mm}{@{}YYY@{}}
  \rowcolor{TableHeading}\color{CoCPaperBox}\bfseries СИЛ + РЗМ & \color{CoCPaperBox}\bfseries Бонус к Урону & \color{CoCPaperBox}\bfseries Комплекция \\
  2-64 & -2 & -2 \\
  65-84 & -1 & -1 \\
  85-124 & Нет & 0 \\
  125-164 & +1D4 & 1 \\
  165-204 & +1D6 & 2 \\
  \end{tabularx}
  \includegraphics[width=\linewidth]{img/bottom.png}
\end{cocpaperbox}
\noindent \textbf{Очки магии (ОМ):} равны 1/5 ВОЛ. Обведите в кружок это значение на листе исследователя. Очки магии используются в колдовстве, а также для подпитки магических устройств и волшебных эффектов. Потраченные очки восстанавливаются сами со скоростью 1 очко в час. Если у персонажа не осталось очков магии, все последующие очки вычитаются из его очков здоровья, любая подобная потеря отражается физическим уроном в том виде, в котором реши Хранитель.
\smallbreak
\noindent \example{ВОЛ 40 предоставляет 8 очков магии. При колдовстве заклятия необходимо потратить 2 очка. Количество имеющихся очков у исследователя временно падает до 6.}
\smallbreak
\noindent \textbf{Показатель удачи (Удача):} определите его, кинув 3D6 и умножьте результат на 5. Обведите итоговое значение на листе исследователя. Бросок на показатель Удачи обычно используется, чтобы определить, насколько расположены в вашу или не вашу пользу внешние обстоятельства. Как и с Рассудком, вы должны выкинуть меньше или равно показателю Удачи, чтобы бросок считался успешным. На стр. 15 приведены подробности о броске на Удачу.
\smallbreak
\noindent \example{Персонаж Романа бежит от толпы зомби и запрыгивает в ближайший автомобиль. Хранитель просит Рому сделать бросок Удачи, чтобы определить, находятся ли ключи в зажигании (т.к. это абсолютно случайная вероятность). Роман делает процентный бросок, кидая 1D100 и выкидывает 28, что меньше его показателя Удачи --- он поворачивает ключ и двигатель заводится!}

\subsection*{Шаг третий: род деятельности и навыки}

К этому моменту у вас уже должно быть представление, чем вам исследователь может зарабатывать себе на жизнь. Помните, что ``исследователь'' не обязательно означает палеонтолога или журналиста. Выбор вида занятости определит навыки, доступные вашему персонажу. Для начала нужно выбрать род деятельности. Все, кого вам было бы интересно отыгрывать, подойдет, но итоговый выбор нужно согласовать с Хранителем. Наиболее популярными видами деятельности в {\it Зове Ктулху} являются Профессор, Журналист, Оккультист и Археолог, но в целом ваш выбор ограничен только фантазией.

Либо выберите вашу профессию из приведенного списка и используйте ее навыки, либо, используя общий список навыков создайте свою собственную. Для этого выберите восемь навыков, которые наиболее подходят для вашей профессии. Эти навыки будут ``профессиональными навыками''.
\smallbreak
\noindent \textbf{Примечание:} краткое описание разных навыков может быть найдено на стр. 10-12.
\smallbreak

Теперь вы можете распределить очки навыков на листе персонажа. Ни один персонаж не может добавлять очки к навыку Знания Мифологии Ктулху, т.к. подразумевается, что любой персонаж ничего не знает об ужасах мироздания.

Распределите следующие значения по восьми профессиональным навыкам, а также по навыку Благосостояния: на один из навыков 70\%, на два 60\%, на три 50\% и еще на три 40\% (установите значения выбранных навыков равные именно этим значениям, не обращайте внимания на базовые значения навыков, указанные на листе персонажа).

\smallbreak

\noindent \example{Полина хочет играть за Журналистку и распределяет следующие навыки: Искусство/Ремесло (Фотография) 50\%, История 40\%, Использование Библиотек 50\%, Родной Язык (Английский) 60\%, Психология 40\%. Для общения она выбирает Убеждение 70\% (очень убедительная!). Затем она смотрит на лист персонажа и видит еще два навыка, которые могут оказаться полезными журналисту: Наблюдательность 50\% и Скрытность 60\%. У нее осталось одно значение - 40\%, которое она решает назначить своему Благосостоянию. Полина записывает эти значения в большую ячейку рядом с каждым навыком.}

\smallbreak

После распределения очков по профессиональным навыкам, выберите ``личные навыки''. Это те навыки, которыми ваш персонаж овладел вне своей основной рабочей деятельности. Выберите четыре не профессиональных навыка и улучшите их на 20\% (прибавьте 20\% к значению, указанному для выбранного навыка на листе персонажа).

Мы рекомендуем записывать значения ваших навыков в том же формате, что и Характеристики -  Обычное / Половинное / Пятое - они понадобятся вам во время игры (на стр 23 приведена Памятка со значениями). Само собой, если вам удобнее, можете высчитать эти значения, когда они вам понадобятся во время игры.

\smallbreak

\noindent \example{Рома выбирает в качестве рода деятельности профессию Солдата. Так как этой профессии нет в этой книге, он сам решает, что ему подойдут следующие навыки: Скалолазание, Уклонение, Схватка (Рукопашный Бой), Огнестрельное Оружие (Винтовка/Дробовик), Первая Помощь, Иностранный Язык, Скрытность и Выживание. Рома распределяет свои очки следующим образом: Скалолазание 60\%, Благосостояние 40\%, Уклонение 60\%, Схватка (Рукопашный Бой) 70\%, Огнестрельное Оружие (Винтовка/Дробовик) 50\%, Первая Помощь 40\%, Иностранный Язык 50\% (он выбирает Испанский в качестве второго языка), Скрытность 50\%, Выживание 40\%.

Затем Роман выбирает четыре личных навыка (прибавляя 20\% к их стартовому значению на листе персонажа): Вождение Автомобиля 40\%, Механика 30\%, Наблюдательность 45\% и Прыжок 40\%. Каждое значение навыка он записывает в ячейку рядом с соответствующим названием и заодно вносит значения для Обычных/Половинных/Пятых, например ``Наблюдательность 45 (22/9).''
}

\begin{cocpaperbox}{}{}
  \raisebox{-10mm}{\includegraphics[width=\linewidth]{img/top.png}}
  \cocpaperboxtitle{Примеры профессий исследователей}
  \textbf{АНТИКВАР}---Оценка, Искусство/Ремесло (Любое), История, Использование Библиотек, Иностранный Язык, один навык для общения (Обаяние, Увещевание, Запугивание или Убеждение), Наблюдательность и один любой навык.
  \bigbreak
  \textbf{ПИСАТЕЛЬ}---Искусство (Литература), История, Использование Библиотек, Знание Природы или Оккультные Знания, Иностранный Язык, Родной Язык, Психология и один любой навык.
  \bigbreak
  \textbf{ВОЛЬНЫЙ ХУДОЖНИК}---Искусство/Ремесло (Любое), Огнестрельное Оружие, Иностранный Язык, Верховая Езда, один навык для общения (Обаяние, Увещевание, Запугивание или Убеждение), любые три других навыка.
  \bigbreak
  \textbf{ДОКТОР МЕДИЦИНЫ}---Первая Помощь, Иностранный Язык (Латынь), Медицина, Психология, Естественные Науки (Биология), Естественные Науки (Фармакология), любые два других навыка как научные или личные специализации (например, психиатр может выбрать Психоланализ).
  \bigbreak
  \textbf{ЖУРНАЛИСТ}---Искусство/Ремесло (Фотография), История, Использование Библиотек, Родной Язык, один навык для общения (Обаяние, Увещевание, Запугивание или Убеждение), Психология, два любых других навыка.
  \bigbreak
  \textbf{СЛЕДОВАТЕЛЬ}---Искусство/Ремесло (Лицедейство) или Маскировка, Огнестрельное Оружие, Юриспруденция, Слух, один навык для общения (Обаяние, Увещевание, Запугивание или Убеждение), Психология, Психология, Наблюдательность, один любой навык.
  \bigbreak
  \textbf{СЫЩИК}---Искусство/Ремесло (Фотография), Маскировка, Юриспруденция, Использование Библиотек, один навык для общения (Обаяние, Увещевание, Запугивание или Убеждение), Психология, Наблюдательность, один любой навык.
  \bigbreak
  \textbf{ПРОФЕССОР}---Использование Библиотек, Иностранный Язык, Родной Язык, Психология, четыре любых других навыка как научные или личные специализации.
  \includegraphics[width=\linewidth]{img/bottom.png}
\end{cocpaperbox}

\subsection*{Шаг четвертый: предыстория}

Взгляните на навыки, характеристики и их значения. Вас захватывает воображение и вы начинаете представлять себе вашего персонажа во плоти. По мере того, как он вырисовывается все отчетливее, вы можете начать описывать его на бумаге. Кто он на самом деле? Где он вырос? Какая у него семья? Чем больше вы думаете о своем персонаже, тем более проработанным он окажется в итоге и тем интереснее будет играть за него в Зове Ктулху.

Каждый пункт жизнеописания (на обратной стороне листа персонажа) должен содержать небольшое краткое утверждение. Подумайте о том, каким персонаж будет казаться людям, встретившим его впервые и запишите ваши мысли в Личном Описании. Во что ваш персонаж верит и как относится к жизни? Опишите одним предложением его Идеологию/Верования. Какие маннеризмы ему присущи? Запишите его неприятные привычки и тому подобные вещи в графе Особенности. Вам не обязательно заполнять все графы в листе - достаточно начать с двух-трёх пунктов, чтобы персонаж не был статистом.
\smallbreak
\noindent \example{Полина записывает в графе Значимые Места ``Родилась и выросла в Нью-Йорке'', в графе Важные Вещи ``Не расстаюсь со своим верным револьвером'', а в графе Идеология/Верования - ``Наука может всему найти объяснение''.}

\subsection*{Шаг пятый: последние штрихи}

У вас уже есть кто-то, напоминающий завершенного персонажа. Взгляните в верхнюю часть листа персонажа и убедитесь, что вы подобрали ему имя, пол и возраст. На обратной стороне листа запишите оборудование, которое может быть у персонажа вашей Профессии.
\smallbreak
\noindent \example{Полина записывает ``Записная книжка, карандаши, перьевая ручка, расческа, заколки'', в графе Оборудование и Вещи, т.к. это то, что вполне может быть при себе у журналиста. Заколки вполне могут пригодиться, когда предстоит открывать замок!}
\smallbreak
На листе также есть пространство, куда вы можете вклеить портрет вашего исследователя. Если вы используете загружаемый PDF, то на слот под портрет можно просто кликнуть и загрузить изображение с компьютера.

Пока не беспокойтесь о графе Наличные и Имущество - она предназначена для продвинутой игры, где сбережения и наличка могут играть решающую роль (подробнее об этом можно прочитать в основной книге правил Зова Ктулху).

\section*{Игровая система}

Во время драматического накала, игра может потребовать ``броска на навык''. Проход по хорошо освещенному коридору не является драматической ситуацией, а вот бег по заваленному всяким мусором проходу, в то время как вас преследует толпа чудовищ - является!

Когда вам нужно совершить бросок на навык, вам нужно договориться с Хранителем о цели броска. Если бросок успешен, вы достигаете заявленной цели. Кроме того, если вы совершаете успешный бросок на навык, вам нужно отметить галочкой этот навык на листе персонажа. Подобную галочку можно поставить у каждого навыка только один раз. В конце сценария, этот навык может увеличиться, т.к. вы получили опыт в его использовании. Подробнее см. Награда за Успех на стр. 22.

Иногда может приходиться совершить бросок, для которого на листе персонажа нет навыка. В таком случае, вам нужно взглянуть на характеристики исследователя и определить, какая из них лучше подходит для данного случая и использовать ее как навык.

\subsection*{Броски на навык и уровни сложности}

Ваш Хранитель скажет вам, когда вам предстоит бросок на навык и то, насколько сложным он будет.
\smallbreak
\begin{itemize}[leftmargin=4mm]
  \item Несложная цель требует результата броска 1D100 равного или меньше вашего значения навыка (Обычный успех).
  \item Сложная цель требует результата равного или меньше половины вашего навыка (Тяжелый успех)
  \item Задача на пределе возможностей человека требует результата броска меньше или равной пятой значения вашего навыка (Экстремальный успех).
\end{itemize}
\smallbreak
Если вы можете обосновать действиями вашего исследователя такую необходимость - вы можете ``протолкнуть'' или ``пушнуть'' (от англ. ``to push'') ваш бросок. Пуш позволяет вам бросить кубики еще раз. Однако, в этот раз ставки выше.

\begin{fullcocpaperbox}{}{}
  \raisebox{-10mm}{\includegraphics[width=\linewidth]{img/top.png}}
  \cocpaperboxtitle{Уровни успеха}
  (худший) ПРОВАЛ --- ОБЫЧНЫЙ УСПЕХ --- ТЯЖЕЛЫЙ УСПЕХ --- ЭКСТРЕМАЛЬНЫЙ УСПЕХ
  \includegraphics[width=\linewidth]{img/bottom.png}
\end{fullcocpaperbox}

Если вас постигнет неудача и в этот раз, Хранитель иожет навлечь ужасные неприятности на голову вашего персонажа. Перед тем, как вы пушнёте вам бросок, Хранитель может предсказать вам то, что случится, если вы провалите второй бросок. После этого игрок может оценить - хочет он пушить или нет.
\smallbreak
\noindent \example{ваш исследователь хочет приподнять тяжелую каменную дверь, преграждающую вход в склеп. Хранитель определяет сложность задачи как Тяжелую и просит вас сделать бросок на СИЛ, сообщая, что требуется ``Тяжелый успех''. Показатель СИЛ вашего персонажа равен 60, так что вам нужно выбросить 30 или меньше. Вы бросаете кубики, но выпадает 43---вы провалили бросок, т.к. выкинули больше половины вашего значения СИЛ. Вы спрашиваете, можно ли пушнуть бросок, говоря, что ваш персонаж удваивает усилия и применяет лопату в качестве рычага. Хранитель позволяет вам совершить второй бросок, но предупреждает, что если вас опять постигнет неудача, что вам не только не удастся открыть дверь, но и ``нечто'' может вас услышать и отправиться по вашему запаху!}

\subsection*{Противостоящие броски на навык}

Если два исследователя противостоят друг другу или один персонаж находится в противостоянии со значительным персонажем под контролем Хранителя, у которого есть собственные значения навыков, Хранитель может попросить совершить ``противостоящий бросок''.

Для того, чтобы его совершить, обе стороны кидают бросок на навык и сравнивают уровни своего успеха. Таким образом, Обычный успех бьет Неудачу, Тяжелый успех бьет Обычный, а Экстремальный - бьет Тяжелый. В случае ничьей побеждает тот, чье значение навыка выше. Если и значения навыка одинаковые - то решает бросок 1D100. Чье значение ниже - тот и выиграл.

\subsubsection*{Кубик Преимущества и Кубик Помехи}

Иногда окружение, условия и/или доступное время могут как помочь, так и помешать броску. При определенных условиях, Хранитель может предоставить ``преимущество'' либо ``помеху'' к броску. Каждая помеха вычитает одно преимущество. Преимущества и помехи действуют подобно увеличенной сложности и могут использоваться вместе с ней или вместо нее. Но, как правило, их используют для противостоящих бросков.
\smallbreak
\smallbreak
\noindent\textbf{За каждый кубик преимущества:} киньте дополнительный 1D10 отвечающий за десятки, а не единицы вместе с кубиками, используемыми в броске. То есть вы должны кинуть три кубика: один с единицами и два с десятками. Чтобы использовать преимущество, выберите лучший показатель на кубиках с десятками (более низкий).
\smallbreak
\noindent \example{два соперничающих исследователя, Малькольм и Хью борются за внимание Леди Грин. Только один из них может рассчитывать на ее благосклонность и руку, так что Хранитель определяет, что для понимания того, кто оказывается более успешным ухажером требуется противостоящий бросок. Игроки и Хранитель договариваются, что наиболее подходящим будет бросок на Обаяние. Хранитель вспоминает все, что произошло в ходе сценария к этому моменту: Малькольм посетил Леди Грин дважды, не забывая о дорогих подарках, в то время как Хью посетил ее лишь единожды и не позаботился о внешнем виде. Хранитель определяет, что Малькольм имеет преимущество и может использовать его в броске.

Игрок, играющий за Хью сперва кидает обычный бросок на навык. Значение его Обаяния равно 55, и игрок выбрасывет 40 на десятках и 5 на единицах, всего 45 - Обычный успех.

Игрок Малькольма также бросает обычный бросок на навык, но с преимуществом. Его Обаяние равно 50. Он кидает три кубика: два на десятки и один на единицы. Кубик с единицами показывает значение 4, а на десятках выпало 20 и 40. Не долго думая, игрок выбирает 20, в сумме 24 - Тяжелый успех.

Малькольм побеждает в противостоянии и его сватовство к Леди Грин принято благосклонно.
}
\smallbreak
\smallbreak
\noindent\textbf{За каждый кубик помехи:} киньте дополнительный 1D10 отвечающий за десятки, со стандартной парой кубиков. То есть вы должны кинуть три кубика: один с единицами и два с десятками. Чтобы получить помеху, вы должны худший показатель на кубиках с десятками (более высокий).
\smallbreak
\noindent \example{из-за череды неудачных стечений обстоятельств, два исследователя - Феликс и Гаррисон - оказались в плену у членов культа Алой Улыбки. Культисты решили ``поразвлечься'' с исследователями и хотят, чтобы они прошли Ордалию Боли, в ходе которой выживет только один. Проигравший будет принесен в жертву омерзительному божеству культа.

Ордалия Боли заключается в том, что нужно как можно дольше продержать тяжелый камень. Это требует противостоящего броска на СИЛ от обоих исследователей. Однако Хранитель определяет, что Гаррисон будет совершать бросок с помехой, ведь он недавно получил тяжелую травму (как раз тогда, когда был захвачен культистами) и все еще не восстановился.

Игрок Феликса выкидывает 51 на своем броске по СИЛ. Его показатель СИЛ равен 60---у него Обычный успех.

Сила Гаррисона - 55. Его игрок выкидывает 20 и 40 на кубиках с десятками и 1 на кубике с единицами. Он должен использовать более высокий результат из-за помехи---в итоге у Гаррисона тоже Обычный успех.

У обоих игроков Обычный успех, но Феликс выигрывает, т.к. его показатель характеристики СИЛ выше. Феликс удержал камень над головой дольше, чем изнемогающий от боли Гаррисон. Культисты визжат от удовольствия и тащат обессиленного Гаррисона к алтарю...
}

\subsubsection*{Броски Удачи}

Хранитель может просить совершить бросок на Удачу когда нужно определить обстоятельства вне контроля исследователей и отыграть слепую руку судьбы. Если, например, исследователь хочет узнать, есть ли рядом предмет, который можно использовать в качестве оружия или остался ли еще заряд в только что найденном фонарике - тогда нужен бросок на Удачу. Обратите внимание - если в ситуации уместнее кинуть навык или характеристику - тогда нужно кидать на них, а не на Удачу. Чтобы бросок на Удачу считался успешным, нудно выкинуть больше либо равно текущему показателю Удачи исследователя.

Если Хранитель просит совершить ``групповой бросок на Удачу'', это означает, что исследователь с худшим (самым низким) показателем Удачи из участвующих в сцене должен бросить кости от лица всей группы.
\smallbreak
\noindent \example{найти кэб - не требует броска кубиков, но найти кэб до того, как подозреваемые скроются на своей машине - уже зависит от броска Удачи. Показатель навыка Благосостояния может сыграть свою роль в привлечении внимания таксиста, который ищет щедрого на чаевые клиента. Но быстро найти такси в два часа утра в нехорошем квартале города уже скорее будет зависеть от Удачи. А вообще, найдется ли кэб? Нет навыка, который отвечал бы за местонахождения кэбов - это целиком воля судьбы, так что требуется бросок Удачи.}

\section*{Рассудок (РАС)}

Когда исследователь сталкивается с необъяснимыми ужасами Мифологии Ктулху или с чем-то естественным, но устрашающим (таким как вид обезображенного трупа друга), нужно прокинуть 1D100 по вашему текущему показателю рассудка. Если вы выбросите больше - вы потеряете большее число очков Рассудка, если меньше - либо вообще не потеряете, либо потеряете меньше. Потеря очков Рассудка обычно описывается в готовых сценариях как ``0/1D6'' или ``2/1D10''. Число слева от слеша говорит о том, сколько вы потеряете в случае успеха, а справа - сколько потеряете в случае неудачи.

Когда вы проваливаете бросок на Рассудок, Хранитель отыгрывает ваше следующее действие самостоятельно, ведь вас охватывает неконтролируемый страх: возможно, вы неожиданно вскрикиваете или невольно нажимаете на курок вашего револьвера.

Если исследователь теряет 5 или больше очков Рассудка в ходе одного броска на Рассудок, это означает, что они получают серьезную эмоциональную травму. Игрок должен кинуть 1D100. Если результат меньше или равен показателю Интеллекта (ИНТ) его персонажа, это означает, что персонаж вполне понял, что перед ним предстало и впадает во временное безумие (длительностью в  1D10 часов). Если игрок провалил бросок, то их разум инстинктивно отказался воспринимать ужас и они остаются (пока что) в здравом уме.
\begin{fullcocpaperbox}{}{}
  \raisebox{-10mm}{\includegraphics[width=\linewidth]{img/top.png}}
  \cocpaperboxtitle{Примеры фобий и маний}
  Существуют сотни возможных фобий и маний. Здесь приведены лишь некоторые из возможных.
  \begin{multicols}{2}
  \subsubsection*{Фобии}
  \begin{itemize}[leftmargin=4mm]
  \item Боязнь высоты (акрофобия).
  \item Боязнь пауков (арахнофобия).
  \item Боязнь книг (библиофобия).
  \item Боязнь зеркал (эйзоптрофобия).
  \item Боязнь крови (гемафобия).
  \item Боязнь мертвых существ (некрофобия).
  \item Боязнь зубов (одонтофобия).
  \item Боязнь огня (пирофобия).
  \item Боязнь телефонов (телефонофобия).
  \item Боязнь незнакомцев или чужаков (ксенофобия).
\end{itemize}
  % \bigbreak
  % \bigbreak
  % \bigbreak  \bigbreak
  % \bigbreak
  % \bigbreak
  \subsubsection*{Мании}
  \begin{itemize}[leftmargin=4mm]
  \item Патологическая вежливость (агатомания).
  \item Одержимость болью (алгомания).
  \item Иррациональная жизнерадостность (аменомания).
  \item Одержимость кражей книг (библиоклептомания).
  \item Одержимость вершением справедливости (дикемания).
  \item Неконтролируемое желание смеяться (гелиомания).
  \item Иррациональное желание кричать (клазомания).
  \item Иррациональная потребность в кражах (клептомания).
  \item Уверенность в наличии воображаемой болезни (носомания).
  \item Иррациональное желание лгать (псевдомания).
\end{itemize}
  \end{multicols}
  \includegraphics[width=\linewidth]{img/bottom.png}
\end{fullcocpaperbox}
В дополнение, безумный исследователь страдает от ``приступа безумия'' --- киньте 1D10 и обратитесь к Таблице Приступов Безумия (стр. 17). Если исследователь находится в компании своих товарищей, то отыгрываете эффект от раунда к раунду. Если исследователь один, то вы можете использовать этот результат, чтобы описать, как его находят через некоторое время в плохом состоянии, возможно запертым изнутри в шкафу, или пьяным вдрызг в канаве.

Если ваш исследователь временно безумен, Хранитель может решить добавить фобию или манию на вашем листе персонажа (такие как ``боязнь темноты'', ``боязнь замкнутых пространств'', ``клептомания''). Как альтернатива, Хранитель также может переписать одну из граф в вашей предыстории, искажая ее (там, где вы написали ``доверчивая'' в качестве Черты, Хранитель может переписать на ``боязливая'').

Пока исследователь находится в состоянии временного безумия, Хранитель может наградить его ``Бредом'' (галлюцинациями), например персонаж не может понять - за его ногу цепляется зомби или это лишь бездомный человек попросил мелочь? В таком случае игрок может быть уверен только если он прокинет ``определение действительности'', для чего нужно опять совершить бросок на Рассудок. Если он успешен, персонаж может понять, что реально, а что нет. Если не успешен - персонаж погружается еще глубже в пучину безумия!

После прохождения 1D10 часов, исследователь приходит в себя и не может быть подвержен галлюцинациям, но измененная предыстория, фобии и мании остаются.

\end{document}