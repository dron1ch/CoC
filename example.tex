\documentclass[letterpaper,twocolumn,openany, twoside, 8pt, usenames]{cocbook}
% \documentclass[twocolumn]{book}

\usepackage[utf8]{inputenc}
\usepackage[english, russian]{babel}
\usepackage{noto}
\usepackage{fontspec}
\usepackage{titlesec}
\usepackage{lipsum}
\definecolor{MetallicGold}{RGB}{138, 5, 9}

% \setmainfont[Ligatures=TeX]{Kurier}
% \newfontfamily\cyrillicfont{Kurier}[Script=Cyrillic]
\newfontfamily\headingfont[]{Oswald-Regular.ttf}
\newfontfamily\smallheadingfont[]{NotoSerif}
\setmainfont[]{NotoSerif}
\titleformat{\section}[hang]
  {\headingfont\huge\bfseries\color{MetallicGold}}
  {\MakeUppercase{\chaptertitlename}\ \thesection:}
  {30pt}
  {\MakeUppercase}
\titleformat{\subsection}[hang]
  {\smallheadingfont\large\bfseries\color{MetallicGold}}
  {\MakeUppercase{\chaptertitlename}\ \thesubsection:}
  {1em}
  {\MakeUppercase}
% \titleformat*{\section}{\LARGE\headingfont}{\MakeUppercase}
% \setmainfont{Oswald-Regular.ttf}

\renewenvironment{quote}{%
   \list{}{%
     \leftmargin0.0cm   % this is the adjusting screw
     \rightmargin\leftmargin
   }
   \item\relax
}
{\endlist}

\let\oldquote\quote
\let\endoldquote\endquote
\renewenvironment{quote}[2][]
  {\if\relax\detokenize{#1}\relax
     \def\quoteauthor{~#2}%
   \fi
   \oldquote}
  {\par\nobreak\smallskip\hfill(\quoteauthor)%
   \endoldquote\addvspace{\bigskipamount}}

\hyphenchar\font=-1
\sloppy

\textwidth=6.29in
\linespread{1.5} 

\begin{document}
\scriptsize

\section*{Что такое зов Ктулху?}
\begin{quote}{---Г.Ф. Лавкрафт}
{\it Самая старая и самая сильная эмоция человечества - это страх. А старейшим и сильнейшим видом страха является страх перед неизвестностью.}
\end{quote}

\begin{quote}{Г.Ф. Лавкрафт, ---{\it Безымянный город}}
{\it То не мертво, что вечность охраняет, Смерть вместе с вечностью порою умирает.}
\end{quote}

{\it Зов Ктулху} - игра о тайнах, загадках и ужасе. Игроки берут на себя роль отважных исследователей и путешествуют в странные и опасные места, раскрывают зловещие заговоры и защищают мир от ужасов ночи. На своем пути они встретят существ, обитающих за пределами пространства и времени, само существование которых угрожает рассудку, омерзительных чудовищ и безумных культистов. Им предстоит отыскивать секреты, которые не должен знать ни один человек, погружаясь в тексты старинных фолиантов. Эти обычные люди встретят множество преград на своем пути, но их героические действия вполне могут определить судьбу мира.

Созданная Сэнди Питерсеном и впервые опубликованная в 1981 году настольная ролевая игра {\it Зов Ктулху} на протяжении более 35 лет определяла жанр и часто называется одной из лучших ролевых игр. Для тех, кто достаточно смел духом чтобы раскрыть ее тайны, эта игра готовит награды за пределами вашего воображения!

В игре, каждый игрок берет на себя роль персонажа, а один из игроков - роль рефери---Хранителя Тайного Знания ("Хранителя"), который служит модератором в игре и представляет игрокам сюжет и сеттинг в рамках которых проходит их приключение. Используя игровые кубики и несложные правила, вы определяете успех или неудачу действий персонажей, в то время как сюжет заставляет их попадать в сложные и опасные ситуации.

Этот буклет позволит вам создать персонажа для настольной ролевой игры {\it Зов Ктулху}, а также обучит основам игровых правил---достаточных для того, чтобы вы с вашими друзьями могли начать играть и насладиться стартовыми приключениями. Полноценные книги правил, а также приключения можно купить на сайте создателей игры - компании Chaosium.

\subsection*{Роли игроков}

В каждой игре игроки берут на себя одну из двух ролей: либо исследователя, либо Хранителя. Большинство берет на себя роль исследователя, потому что исследование - это основное, чем они будут заниматься: решить загадку или найти выход из сложной ситуации. Сюжеты в которых будут участвовать персонажи специально будут стараться бросить им вызов: они могут оказаться ранены, испытать неподвластные рассудку видения или даже могут быть проглочены чудовищем! По мере того, как игра будет развиваться, исследователи смогут узнать о странной магии и ужасных и чуждых созданиях, получить тайные знания из магических гримуаров, полных зловещих тайн и улучшить свои навыки, по мере того, как они становятся более опытными и готовыми к приключениям.

Один игрок берет на себя роль Хранителя. Он выбирает сценарий для игры или может самостоятельно подготовить свой собственный оригинальный сюжет. В игре, Хранитель подготавливает сцену, описывает происходящее и изображает людей, которых встречают исследователи (они обычно называются NPC, от английского "Non-Player Character" - персонаж, роль которого принадлежит Хранителю). Хранитель также помогает определять результат действий и служит арбитром в части игровых правил. Так как Хранитель должен тратить дополнительное время и силы на подготовку к игре, игроки часто меняют роль Хранителя от сценария к сценарию. О Хранителе можно думать как о режиссере, который ставит фильм, актеры в котором (игроки) не знают, что будет дальше по сюжету.

Сам процесс игры представляет собой развивающееся взаимодействие между игроками---в образе их персонажей, разгадывающих тайну---и Хранителем, который представляет мир в котором представлена эта тайна.

В этой игре нет никакого игрового поля. Процесс, как правило, --- простой разговор: Хранитель представляет какую-то ситуацию, а затем игроки, от лица своих персонажей, говорят о том, что они хотят сделать.

Используя систему правил, чтобы все было честно и последовательно, Хранитель сообщает им, могут ли они совершить то, что хотят и если да, что нужно сделать, чтобы добиться желаемого эффекта. Как правило, это подразумевает бросок кубиков. Кубики позволяют определить результат событий и разнообразных ситуаций, а также гарантируют непредвзятость происходящего, а также добавить драмы и накала страстей --- результат броска может означать непредвиденный сюрприз, жестокое поражение или избежание смерти буквально на волосок! После того как результат броска определен, Хранитель описывает, что происходит, спрашивая у игроков, как реагируют их персонажи и так далее.

Цель ролевой игры - повеселиться. Даже когда сердца бьются, а со лба капает пот - людям нравиться пугаться, если только ужасы не настоящие. Для некоторых, расслабление после пережитого страха --- само по себе желанный результат. Другим просто нравится пребывать в этом состоянии. {\it Зов Ктулху} сможет предоставить всем причастным и то и другое!



\end{document}
