% PATH=/usr/local/texlive/2019/bin/x86_64-linux:$PATH
% lualatex example.tex 

\documentclass[letterpaper,twocolumn,openany, twoside, 11pt, usenames]{cocbook}

\usepackage[utf8]{inputenc}
\usepackage[english, russian]{babel}
\usepackage{noto}
\usepackage{fontspec}
\usepackage{titlesec}
\usepackage{lipsum}
\usepackage{hyphenat}
\usepackage{graphicx}
\usepackage{enumitem}
\usepackage[table]{xcolor}
\usepackage{tabularx}
\usepackage{multicol}
\usepackage{cuted}


\newfontfamily\headingfont[]{Oswald-Regular.ttf}
\newfontfamily\smallheadingfont[]{NotoSerif}
\setmainfont[]{NotoSerif}
\titleformat{\section}
  {\headingfont\fontsize{20}{1em}\bfseries\color{RedCthulhu}}
  {\MakeUppercase{\chaptertitlename} \thesection:}
  {1em}
  {\MakeUppercase}

\titleformat{\subsection}[hang]
  {\smallheadingfont\fontsize{14}{1em}\bfseries\color{RedCthulhu}}
  {\MakeUppercase{\chaptertitlename}\ \thesubsection:}
  {1em}
  {\MakeUppercase}
\titleformat*{\subsubsection}{\normalfont\fontsize{12}{1em}\bfseries\color{RedCthulhu}}

\renewenvironment{quote}{%
   \list{}{%
     \leftmargin0.0cm   % this is the adjusting screw
     \rightmargin\leftmargin
   }
   \item\relax
}
{\endlist}

\let\oldquote\quote
\let\endoldquote\endquote
\renewenvironment{quote}[2][]
  {\if\relax\detokenize{#1}\relax
     \def\quoteauthor{~#2}%
   \fi
   \oldquote}
  {\par\nobreak\smallskip\hfill(\quoteauthor)%
   \endoldquote\addvspace{\bigskipamount}}

\hyphenchar\font=-1
\sloppy

\newcolumntype{Y}{>{\centering\arraybackslash}X}

\textwidth=6.70in
\linespread{1.5} 

\definecolor{cthulhuGreen}{RGB}{68, 83, 38}

\newlength{\seplinewidth}
\newlength{\seplinesep}
\setlength{\seplinewidth}{1mm}
\setlength{\seplinesep}{2mm}
\colorlet{sepline}{cthulhuGreen}
\newcommand*{\sepline}{%
  \par
  \vspace{\dimexpr\seplinesep+.5\parskip}%
  \cleaders\vbox{%
    \begingroup % because of color
      \color{sepline}%
      \hrule width\linewidth height\seplinewidth
    \endgroup
  }\vskip\seplinewidth
  \vspace{\dimexpr\seplinesep-.5\parskip}%
}

\begin{document}
\scriptsize

\section*{\nohyphens{Что такое зов Ктулху?}}
\begin{quote}{--- Г.Ф. Лавкрафт}
{\it Самая старая и самая сильная эмоция человечества --- это страх. А старейшим и сильнейшим видом страха является страх перед неизвестностью.}
\end{quote}

\begin{quote}{Г.Ф. Лавкрафт, --- {\it Безымянный город}}
{\it То не мертво, что вечность охраняет, Смерть вместе с вечностью порою умирает.}
\end{quote}

{\it Зов Ктулху} --- игра о тайнах, загадках и ужасе. Игроки берут на себя роль отважных исследователей и путешествуют в странные и опасные места, раскрывая зловещие заговоры и защищая мир от ужасов ночи. На своём пути они встретят существ, обитающих за пределами пространства и времени, само существование которых угрожает рассудку, омерзительных чудовищ и безумных культистов. Погружаясь в тексты старинных фолиантов, им предстоит раскрывать секреты, которые не должен знать ни один человек. Они встретят множество преград на своем пути, но их героические действия вполне могут определить судьбу мира.

Созданная Сэнди Питерсеном и впервые опубликованная в 1981 году настольная ролевая игра {\it Зов Ктулху} на протяжении более 35 лет определяла жанр и часто называется одной из лучших ролевых игр. Тем, кто достаточно смел духом, чтобы раскрыть её тайны, эта игра готовит награды за пределами любых ожиданий!

В игре каждый игрок берет на себя роль персонажа, а один из игроков - роль рефери --- Хранителя Тайного Знания (``Хранителя''), который служит в игре модератором и обрисовывает игрокам сюжет и сеттинг, в рамках которых проходит их приключение. Используя игровые кубики и несложные правила, вы определяете успех или неудачу действий персонажей, в то время как сюжет заставляет их попадать в сложные и опасные ситуации.

Этот буклет позволит вам создать персонажа для настольной ролевой игры {\it Зов Ктулху}, а также обучит основам игровых правил, достаточных для того, чтобы вы с вашими друзьями могли начать играть и насладиться стартовыми приключениями. Полноценные книги правил, а также приключения можно купить на сайте создателей игры --- компании Chaosium.

\subsection*{\nohyphens{Роли игроков}}

В каждой игре игроки берут на себя одну из двух ролей: исследователя либо Хранителя. Большинство выбирает роль исследователя, и исследование --- это основное, чем они будут заниматься: решать загадки или искать выход из сложных ситуаций. Сюжеты будут бросать персонажам вызов: они могут оказаться ранеными, испытать неподвластные рассудку видения или даже быть проглоченными чудовищем! По мере того, как игра будет развиваться, исследователи смогут узнать о странной магии и ужасных, чуждых созданиях, получить тайные знания из магических гримуаров, полных зловещих тайн, и улучшить свои навыки, становясь более опытными и готовыми к приключениям.

Один из игрокок берет на себя роль Хранителя. Он выбирает сценарий для игры или может придумать собственный оригинальный сюжет. В игре Хранитель подготавливает сцену, описывает происходящее и изображает людей, которые попадаются исследователям на пути (они обычно называются NPC, от английского ``Non-Player Character'' --- персонаж, роль которого исполняет Хранитель). Хранитель также помогает определять результат действий и служит арбитром в части игровых правил. Так как Хранитель должен тратить дополнительное время и силы на подготовку к игре, роль Хранителя от сценария к сценарию переходит от игрока к игроку. О Хранителе можно думать как о режиссёре фильма, актёры которого (игроки) не знают, что будет дальше по сюжету.

Сам процесс игры представляет собой развивающееся взаимодействие между игроками --- в образе их персонажей, разгадывающих тайну, --- и Хранителем, который описывает мир этой тайны.

В этой игре нет никакого игрового поля. Процесс представляет собой простой разговор: Хранитель моделирует некую ситуацию, а затем игроки от лица своих персонажей рассказывают о том, что они хотят сделать.

Используя систему правил, чтобы всё было честно и последовательно, Хранитель сообщает им, могут ли они совершить то, что хотят, и если да, что им нужно сделать, чтобы добиться желаемого эффекта. Обычно это подразумевает бросок кубиков, который позволяет определить результат событий, а также гарантирует непредвзятость происходящего и добавляет накала страстей. Результат броска может привести к непредвиденному сюрпризу, жестокому поражению или, напротив, позволить персонажу избежать смерти в последнюю секунду! После того, как результат броска определен, Хранитель описывает происходящее и спрашивает у игроков, как реагируют на это их персонажи. Так и проходит игра.

Цель ролевой игры - повеселиться. Даже когда сердца бьются, а со лба капает пот, людям нравиться пугаться, если только ужасы не настоящие. Для некоторых расслабление после пережитого страха --- само по себе желанный результат. Другим просто нравится пребывать в этом состоянии. {\it Зов Ктулху} сможет порадовать и тех, и других!

\subsection*{\nohyphens{Игровые концепции}}

{\it Зов Ктулху} основан на историях писателя Говарда Филлипса Лавкрафта (20 августа 1890 - 15 марта 1937), чьи истории имели в своей основе оригинальную мрачную философию, согласно которой, мироздание безжалостно и им правят чуждые  человеку и обладающие богоподобной мощью создания. Эта концепция была достаточно новаторской, чтобы увлечь поколения писателей, развивших и расширивших лавкрафтовскую вселенную, которая получила название ``Мифология Ктулху''.

В течение жизни Лавкрафт был практически неизвестен и публиковался исключительно в дешёвых бульварных журналах. Он умер в нищете, но в наше время считается одним из наиболее влиятельных авторов, творивших в жанре ужасов. Несмотря на его устаревшие и во многом расистские взгляды, мы можем вдохновляться творчеством Лавкрафта и использовать его как основу для ролевых игр. Его произведения написаны тяжелым, архаичным языком, как будто сошедшим со страниц тех древних книг, которые часто упоминаются в его рассказах. Однако мы можем заимствовать из его творчества не только ужасы, но и юмористическую сторону: Лавкрафт зачастую подтрунивал сам над собой и над своими коллегами по перу, пародируя их стиль. Современные авторы книг и игр используют творчество Лавкрафта и других его современников, вплетая в повествование как современные, так и исторические проблемы.

Если вы хотите узнать больше о Лавкрафте и Мифологии Ктулху, вашим лучшим помощником станет интернет. Начать можно с Википедии.

\subsubsection*{\nohyphens{Мифология Ктулху}}

Авторство термина ``Мифология Ктулху'', как правило, присваивают Огасту Дерлету --- писателю и раннему поклоннику творчества Лавкрафта. Он стал основателем издательского дома ``Arkham House'', который поставил своей целью сделать широко доступными книги Лавкрафта. Сегодня под этим термином подразумевается вымышленная космология, которая включает богов и чудовищ, тайные знания и то, что называют ``лафкрафтовскими ужасами''. В этой космологии человечеству отведена невзрачная роль быть на задворках вселенной.

Лавкрафт писал: ``Все мои рассказы основаны на предположении, что обычные человеческие законы, интересы и эмоции не имеют никакого значения в контексте мироздания''. Он также полагал, что основополагающие истины вселенной настолько чужды и ужасающи для человеческого восприятия, что малейшее их воздействие на человека может привести к потере разума. И хотя человечество хочет одновременно и покоя, и истины, в космологии Лавкрафта можно обладать только чем-то одним. Человеческий разум --- несовершенное вместилище и не может одновременно содержать космические истины и неповрежденный рассудок.

В Мифологии Ктулху те люди, что жаждут власти, могут избрать путь во тьму и полностью лишиться рассудка в обмен на умение повелевать тайнами времени и пространства. После заключения сделки с дьяволом эти беспощадные чародеи в обмен на ещё большую власть и знания способны вызвать разрушения, которых этот мир ещё не видел. Чуждые существа Мифологии безразличны к человечеству, которое часто воспринимает их как божеств и основывает культы, почитающие их. Подобные культисты --- одни из основных антагонистов в {\it Зове Ктулху}.

\subsubsection*{\nohyphens{Тайна и Исследование}}

Приключения в {\it Зове Ктулху} обычно разворачиваются вокруг тайны, а к персонажам-исследователям обращаются, чтобы найти истину. Зачастую загадка бывает порождена преступлениями безумного идолопоклонника или культа одного из божеств Мифологии. Задача исследователей --- найти улики, каждая из их которых укажет на ответы на стоящие перед персонажами вопросы или задаст им новые. По мере того, как исследователи находят все большее количество улик, их понимание ситуации улучшается, пока они не найдут разгадку, которая может привести к изначальному виновнику и финальному противостоянию. Задача игроков, которые отыгрывают этих исследователей, --- раскрыть тайну и определиться с тем, как они привлекут виновных к ответственности или расправятся с ними. Иногда тайна может быть результатом странного колдовства, действий омерзительного чудовища или другого необычного события, каждое из которых может оказать непредвиденный эффект на исследователей.

Само собой, все приключения разные, и не все начинаются с тайны. Иногда сценарий открывается сценой, в которой персонажи находятся в центре опасной ситуации, из которой им надо выбраться. Каждое приключение, как правило, --- небольшая история, и все эти истории отличаются друг от друга. Истории могут быть связаны друг с другом и формировать более масштабный сюжет, который называют кампанией.

\section*{\nohyphens{Создание исследователя}}

Для того, чтобы играть в {\it Зов Ктулху}, вам необходимо создать персонажа-исследователя. Создание персонажа можно представить в виде формальных шагов, они описаны ниже. Игроки записывают информацию о своих персонажах на специальных листах персонажей, которые содержат всю информацию, необходимую для игры. Их можно скачать здесь: https://www.chaosium.com/cthulhu-character-sheets.
Представленный ниже процесс создания персонажа упрощён, чтобы дать вам возможность начать играть как можно быстрее. Более подробный процесс описан в {\it Книге Хранителя} и {\it Книге Исследователя}.

%\begin{figure}
%  \includegraphics[width=\linewidth]{img/cthulhu.jpg}
%  \caption{A boat.}
%  \label{fig:boat1}
%\end{figure}

%\begin{figure}
%\setthemecolor[CoCPaperBox]
%\end{figure}

%\sepline

\subsection*{Шаг первый: характеристики исследователя}
Итак, исследователь в {\it Зове Ктулху} имеет восемь характеристик:

\begin{itemize}[leftmargin=4mm]
  \item Сила (СИЛ) служит мерилом физической мощи вашего исследователя.
  \item Телосложение (ТЛС) определяет стойкость персонажа и его здоровье.
  \item Сила Воли (ВОЛ) --- это сочетание силы духа, воли и ментальной стабильности.
  \item Ловкость (ЛВК) --- мерило физической ловкости и скорости.
  \item Внешность (ВНШ) отвечает за физическую привлекательность вашего персонажа.
  \item Размер (РЗМ) отражает сочетание высоты и веса исследователя.
  \item Интеллект (ИНТ) --- степень хитроумия вашего персонажа и его способности делать логические и интуитивные выводы.
  \item Образование (ОБР) --- аккумулированный жизненный опыт персонажа, будь то результат образования или жизни на улице.
\end{itemize}

Распределите следующие значения, как вам угодно, среди восьми характеристик: 40, 50, 50, 50, 60, 60, 70, 80. Каждое из этих значение --- это процент, т.е. если вы определяете силу персонажа в 70, это означает, что у них ``СИЛ 70\%'' --- показатель выше среднего, т.е. персонаж довольно силен. Запишите эти значения в большой ячейке рядом с каждой из характеристик. Эти значения называются ``Обычными''.

\subsubsection*{\nohyphens{Половинные и Пятые Значения Характеристик}}

Взглянув на лист исследователя, вы увидите три ячейки рядом с каждой характеристикой: одну большую ячейку (в которой вы только что записали числа для этой характеристики) и ещё две маленькие ячейки сбоку. Верхняя маленькая ячейка для ``Половинного'' значения, та что ниже --- для ``Пятого''.

\begin{itemize}[leftmargin=4mm]
  \item Для того, чтобы использовать Половинное значение характеристики, просто разделите пополам Обычное значение этой характеристики, округляя в меньшую сторону до ближайшего целого значения, если необходимо. Например, если вы выбрали значение СИЛ 70, то Половинное значение (используемое для Сложных бросков) будет равно 35.
  \item Для получения Пятого значения разделите Обычное значение на 5, также округляя в меньшую сторону. Например, для СИЛ 70 Пятое значение (используемое для Экстремально сложных бросков) будет равно 14.
\end{itemize}

Запишите Половинные и Пятые значения для каждой характеристики в лист исследователя.

\subsection*{Шаг второй: вторичные атрибуты}

Имеется некое число атрибутов, определяемых после того, как вы выбрали значения характеристик вашего исследователя. Это Бонус к Урону, Комплекция, Очки Здоровья, Скорость Передвижения, Рассудок и Очки Магии. Кроме того, вам нужно определить показатель Удачи.
\smallbreak
\noindent \textbf{Бонус к Урону (БУ) и Комплекция:} Бонус к Урону определяет, как много дополнительного урона ваш исследователь наносит, когда он совершает успешную атаку в ближнем бою (рукопашном или с применением оружия). Комплекция --- это результат сочетания Размера и Силы, который обычно используется в ``рукопашных маневрах'' (см. стр. 12). Сложите ваши Обычные значения СИЛ и РЗМ и обратитесь к таблице на этой странице. На листе исследователя имеются ячейки для итогового значения Комплекции и Бонуса к Урону.
\smallbreak
\noindent \textbf{Очки Здоровья (ОЗ):} являются результатом суммы РЗМ и ТЛС, разделенной на 10 и округленной в меньшую сторону до ближайшего целого числа. По мере того, как ваш исследователь получает урон в бою или от каких-либо других воздействий, это значение будет уменьшаться.
\smallbreak
\noindent \example{РЗМ 50 и ТЛС 50 в сумме дают 100. Разделенное на 10, это значение равно 10 Очкам Здоровья. Исследователь может получить 10 очков урона, после чего он потеряет сознание и, вполне вероятно, умрет.}
\smallbreak
\noindent \textbf{Скорость Перемещения (ПРМ):} все персонажи-люди перемещаются со Скоростью, равной 8.
\smallbreak
\noindent \textbf{Очки Рассудка (РАС):} изначально равны показателю ВОЛ персонажа. Вы увидите раздел с Очками Рассудка на листе персонажа: вам нужно обвести кружком нужное значение. Это значение используется в процентном (1D100) броске, который применяется чтобы понять, насколько ваш персонаж может сохранять хладнокровие перед лицом кошмара. По мере того, как вы сталкиваетесь с ужасами Мифологии Ктулху, ваш показатель рассудка будет изменяться.
\smallbreak
\noindent \example{ВОЛ 40 дает показатель рассудка, равный 40. Когда вы делаете бросок на Рассудок, вам нужно выкинуть меньше или равно 40 на 1D100, чтобы бросок считался успешным. Значение 41 или выше означает, что бросок на рассудок провалился.}
\smallbreak
\begin{cocpaperbox}{}{}
  \raisebox{-10mm}

  \sepline
  \cocpaperboxtitle{Таблица Бонуса к Урону и Комплекции}
  \noindent \example{Рома выбирает СИЛ, равную 60, и РЗМ, равный 70, в сумме дающие 130. Когда он совершает успешную физическую атаку, он наносит дополнительные 1D4 очка урона (Бонус к Урону). Его Комплекция равна 1.}
  \bigbreak
  \rowcolors{2}{}{TableRow}
  \begin{tabularx}{74mm}{@{}YYY@{}}
  \rowcolor{TableHeading}\color{CoCPaperBox}\bfseries СИЛ + РЗМ & \color{CoCPaperBox}\bfseries Бонус к Урону & \color{CoCPaperBox}\bfseries Комплекция \\
  2-64 & -2 & -2 \\
  65-84 & -1 & -1 \\
  85-124 & Нет & 0 \\
  125-164 & +1D4 & 1 \\
  165-204 & +1D6 & 2 \\
  \end{tabularx}
  \sepline
\end{cocpaperbox}
\noindent \textbf{Очки Магии (ОМ):} равны 1/5 ВОЛ. Обведите в кружок это значение на листе исследователя. Очки Магии используются в колдовстве, а также для подпитки магических устройств и сверхъестественных эффектов. Потраченные очки восстанавливаются сами со скоростью 1 очко в час. Если у персонажа не осталось Очков Магии, все последующие очки вычитаются из его Очков Здоровья. Любая подобная потеря отражается физическим уроном в том виде, в котором решит Хранитель.
\smallbreak
\noindent \example{ВОЛ 40 предоставляет 8 Очков Магии. При колдовстве заклятия необходимо потратить 2 очка. Количество имеющихся очков у исследователя временно падает до 6.}
\smallbreak
\noindent \textbf{Показатель Удачи (Удача):} определите его, кинув 3D6, и умножьте результат на 5. Запишите итоговое значение на листе исследователя. Бросок на показатель Удачи обычно используется, чтобы определить, насколько расположены в вашу или не вашу пользу внешние обстоятельства. Как и в случае с Рассудком, вы должны выкинуть меньше или равно показателю Удачи, чтобы бросок считался успешным. На стр. 9 приведены подробности о броске на Удачу.
\smallbreak
\noindent \example{Персонаж Романа бежит от толпы зомби и запрыгивает в ближайший автомобиль. Хранитель просит Рому сделать бросок Удачи, чтобы определить, находятся ли ключи в зажигании (т.к. это абсолютно случайная вероятность). Роман делает процентный бросок, кидая 1D100 с результатом 28, что меньше его показателя Удачи, --- он поворачивает ключ, и двигатель заводится!}

\subsection*{Шаг третий: род деятельности и навыки}

К этому моменту у вас уже должно быть представление, чем ваш исследователь может зарабатывать себе на жизнь. Помните, что ``исследователь'' не обязательно означает палеонтолога или журналиста. Вид занятости определит навыки, доступные вашему персонажу. Для начала нужно выбрать род деятельности. Подойдет любой, который вам было бы интересно отыгрывать, но итоговое решение нужно согласовать с Хранителем. Наиболее популярными видами деятельности в {\it Зове Ктулху} являются Профессор, Журналист, Оккультист и Археолог, но в целом выбор ограничен только вашей фантазией.

Либо выберите профессию из приведенного списка и используйте её навыки, либо на основе общего списка навыков создайте свою собственную. Для этого выберите восемь навыков, которые наиболее подходят для вашей профессии. Эти навыки называются ``профессиональными навыками''.
\smallbreak
\noindent \textbf{Примечание:} краткое описание разных навыков может быть найдено в конце этого буклета.
\smallbreak

Теперь вы можете распределить очки навыков на листе персонажа. Ни один персонаж не может добавлять очки к навыку Мифология Ктулху, т.к. подразумевается, что любой персонаж ничего не знает об ужасах мироздания.

Распределите следующие значения по восьми профессиональным навыкам и навыку Благосостояния: на один из навыков придется 70\%, на два --- 60\%, на три --- 50\% и ещё на три --- 40\% (установите значения выбранных навыков равными именно этим значениям, не обращайте внимания на базовые значения навыков, указанные на листе персонажа).

\smallbreak

\noindent \example{Полина хочет играть за Журналистку и распределяет следующие навыки: Искусство/Ремесло (Фотография) 50\%, История 40\%, Использование Библиотек 50\%, Родной Язык (Английский) 60\%, Психология 40\%. Для общения она выбирает Убеждение 70\% (очень убедительная!). Затем она смотрит на лист персонажа и видит ещё два навыка, которые могут оказаться полезными журналисту: Наблюдательность 50\% и Скрытность 60\%. У нее осталось одно значение --- 40\%, которое она решает назначить своему Благосостоянию. Полина записывает эти значения в большую ячейку рядом с каждым навыком.}

\smallbreak

После распределения очков по профессиональным навыкам, выберите ``личные навыки''. Это те навыки, которыми ваш персонаж овладел вне своей основной рабочей деятельности. Выберите четыре не профессиональных навыка и улучшите их на 20\% (прибавьте 20\% к значению, указанному для выбранного навыка на листе персонажа).

Рекомендуется записывать значения ваших навыков в том же формате, что и Характеристики, ---  Обычное / Половинное / Пятое --- они понадобятся вам во время игры. Само собой, если вам удобнее, можете высчитать эти значения, когда они вам понадобятся во время игры.

\smallbreak

\noindent \example{Рома выбирает в качестве рода деятельности профессию Солдата. Так как данной профессии нет в этой книге, он сам решает, что ему подойдут следующие навыки: Скалолазание, Уклонение, Схватка (Рукопашный Бой), Огнестрельное Оружие (Винтовка/Дробовик), Первая Помощь, Язык (Иной), Скрытность и Выживание. Рома распределяет свои очки следующим образом: Скалолазание 60\%, Благосостояние 40\%, Уклонение 60\%, Схватка (Рукопашный Бой) 70\%, Огнестрельное Оружие (Винтовка/Дробовик) 50\%, Первая Помощь 40\%, Язык (Иной) 50\% (он выбирает Испанский в качестве второго языка), Скрытность 50\%, Выживание 40\%.

Затем Роман выбирает четыре личных навыка (прибавляя 20\% к их стартовому значению на листе персонажа): Вождение Автомобиля 40\%, Механика 30\%, Наблюдательность 45\% и Прыжок 40\%. Каждое значение навыка он записывает в ячейку рядом с соответствующим названием и заодно вносит значения для Обычных/Половинных/Пятых, например: ``Наблюдательность 45 (22/9).''
}
\newpage
\begin{cocpaperbox}{}{}
  \raisebox{-10mm}
  
  \sepline
  \cocpaperboxtitle{Примеры профессий исследователей}
  \textbf{АНТИКВАР} --- Оценка, Искусство/Ремесло (Любое), История, Использование Библиотек, Язык (Иной), один навык для общения (Обаяние, Увещевание, Запугивание или Убеждение), Наблюдательность и один любой навык.
  \bigbreak
  \textbf{ПИСАТЕЛЬ} --- Искусство (Литература), История, Использование Библиотек, Знание Природы или Оккультные Знания, Язык (Иной), Родной Язык, Психология и один любой навык.
  \bigbreak
  \textbf{ВОЛЬНЫЙ ХУДОЖНИК} --- Искусство/Ремесло (Любое), Огнестрельное Оружие,  Язык (Иной), Верховая Езда, один навык для общения (Обаяние, Увещевание, Запугивание или Убеждение), любые три других навыка.
  \bigbreak
  \textbf{ДОКТОР МЕДИЦИНЫ} --- Первая Помощь, Язык (Иной: Латынь), Медицина, Психология, Естественные Науки (Биология), Естественные Науки (Фармакология), любые два других навыка как научные или личные специализации (например, психиатр может выбрать Психоанализ).
  \bigbreak
  \textbf{ЖУРНАЛИСТ} --- Искусство/Ремесло (Фотография), История, Использование Библиотек, Родной Язык, один навык для общения (Обаяние, Увещевание, Запугивание или Убеждение), Психология, два любых других навыка.
  \bigbreak
  \textbf{СЛЕДОВАТЕЛЬ} --- Искусство/Ремесло (Лицедейство) или Маскировка, Огнестрельное Оружие, Право, Слух, один навык для общения (Обаяние, Увещевание, Запугивание или Убеждение), Психология, Психология, Наблюдательность, один любой навык.
  \bigbreak
  \textbf{СЫЩИК} --- Искусство/Ремесло (Фотография), Маскировка, Право, Использование Библиотек, один навык для общения (Обаяние, Увещевание, Запугивание или Убеждение), Психология, Наблюдательность, один любой навык.
  \bigbreak
  \textbf{ПРОФЕССОР} --- Использование Библиотек, Язык (Иной), Родной Язык, Психология, четыре любых других навыка как научные или личные специализации.
  \sepline
\end{cocpaperbox}

\subsection*{Шаг четвертый: жизнеописание}

Взгляните на навыки, характеристики и их значения. Вас захватывает воображение, и вы начинаете представлять себе вашего персонажа во плоти. По мере того, как он вырисовывается всё отчетливее, вы можете начать описывать его на бумаге. Кто он на самом деле? Где он вырос? Какая у него семья? Чем больше вы думаете о своем персонаже, тем более проработанным он окажется в итоге и тем интереснее будет играть за него в {\it Зове Ктулху}.

Каждый пункт жизнеописания (на обратной стороне листа персонажа) должен содержать небольшое краткое утверждение. Подумайте о том, каким персонаж будет казаться людям, встретившим его впервые, и запишите ваши мысли в ``Личном описании''. Во что ваш персонаж верит и как относится к жизни? Опишите одним предложением его ``Идеологию/Верования''. Какие манеры ему присущи? Отметьте его неприятные привычки и тому подобные вещи в графе ``Черты''. Вам не обязательно заполнять все графы на листе --- достаточно начать с двух-трёх пунктов, чтобы персонаж не был статистом.
\smallbreak
\noindent \example{Полина записывает в графе ``Значимые места'': ``Родилась и выросла в Нью-Йорке'', в графе ``Дорогие сердцу вещи'': ``Не расстаюсь со своим верным револьвером'', а в графе ``Идеология/Верования'': ``Наука может всему найти объяснение''.}

\subsection*{Шаг пятый: последние штрихи}

У вас уже есть кто-то, напоминающий завершенного персонажа. Взгляните в верхнюю часть листа персонажа и убедитесь, что вы подобрали ему имя, пол и возраст. На обратной стороне листа запишите снаряжение, которое может быть у персонажа вашей Профессии.
\smallbreak
\noindent \example{Полина записывает: ``Записная книжка, карандаши, перьевая ручка, расческа, заколки'' в графе Снаряжение и Имущество, т.к. это то, что вполне может быть при себе у журналиста. Заколки могут пригодиться, когда придётся взламывать замок!}
\smallbreak
На листе также есть пространство, куда вы можете вклеить портрет вашего исследователя. Если вы используете загружаемый PDF, то на слот под портрет можно просто кликнуть и загрузить изображение с компьютера.

Пока не беспокойтесь о графе Деньги и Активы --- она предназначена для продвинутой игры, где сбережения и наличные средства могут играть решающую роль (подробнее об этом можно прочитать в основной книге правил {\it Зова Ктулху}).

\section*{Игровая система}

Во время драматического накала игра может потребовать ``броска на навык''. Проход по хорошо освещённому коридору не является драматической ситуацией, а вот бег по заваленному всяким мусором проходу, в то время как вас преследует толпа чудовищ, --- является!

Когда вам нужно совершить бросок на навык, вам нужно договориться с Хранителем о цели броска. Если бросок успешен, вы достигаете заявленной цели. Кроме того, если вы совершаете успешный бросок на навык, вам нужно отметить галочкой этот навык на листе персонажа. Подобную галочку можно поставить у каждого навыка только один раз. В конце сценария этот навык может увеличиться, т.к. вы получили опыт в его использовании. Подробнее см. Награда за Успех на стр. 13-14.

Иногда приходится совершать бросок, для которого на листе персонажа нет навыка. В таком случае вам нужно взглянуть на характеристики исследователя и определить, какая из них лучше подходит для данного случая, и использовать её в качестве навыка.

\subsection*{Броски на навык и уровни сложности}

Ваш Хранитель сообщит вам, когда вам предстоит совершить бросок на навык, и то, насколько сложным он будет.
\smallbreak
\begin{itemize}[leftmargin=4mm]
  \item Несложная цель требует результата броска 1D100, равного или меньше вашего значения навыка (Обычный успех).
  \item Сложная цель требует результата, равного или меньше половины вашего навыка (Трудный успех).
  \item Задача на пределе возможностей человека требует результата броска, равного или меньше пятой значения вашего навыка (Экстремальный успех).
\end{itemize}
\smallbreak
Если вы можете обосновать действиями вашего исследователя такую необходимость, вы можете ``протолкнуть'' или ``пушнуть'' (от англ. ``to push'') ваш бросок. Пуш позволяет вам бросить кубики ещё раз. Однако, в этот раз ставки будут выше.

% \begin{strip}
\begin{fullcocpaperbox}{}{}
  \raisebox{-10mm}
  
  \sepline
  \cocpaperboxtitle{Уровни успеха}
  (худший) ПРОВАЛ --- ОБЫЧНЫЙ УСПЕХ --- ТРУДНЫЙ УСПЕХ --- ЭКСТРЕМАЛЬНЫЙ УСПЕХ
  \sepline
\end{fullcocpaperbox}
% \end{strip}

Если вас постигнет неудача и в этот раз, Хранитель может навлечь ужасные неприятности на голову вашего персонажа. Перед тем, как вы пушнёте бросок, Хранитель может предсказать вам, что случится, если вы провалите второй бросок. После этого игрок может оценить --- хочет он пушить или нет.
\smallbreak
\noindent \example{ваш исследователь хочет приподнять тяжёлый каменный блок, преграждающий вход в склеп. Хранитель определяет сложность задачи как Трудную и просит вас сделать бросок на СИЛ, сообщая, что требуется ``Трудный успех''. Показатель СИЛ вашего персонажа равен 60, так что вам нужно выбросить 30 или меньше. Вы бросаете кубики, но выпадает 43 --- вы провалили бросок, т.к. выкинули больше половины вашего значения СИЛ. Вы спрашиваете, можно ли пушнуть бросок, говоря, что ваш персонаж удваивает усилия и применяет лопату в качестве рычага. Хранитель позволяет вам совершить второй бросок, но предупреждает, что если вас опять постигнет неудача, вам не только не удастся приподнять блок, но и ``нечто'' может вас услышать и отправиться по вашему запаху!}

\subsection*{Противостоящие броски на навык}

Если два исследователя противостоят друг другу или один персонаж находится в противостоянии со значительным персонажем под контролем Хранителя, у которого есть собственные значения навыков, Хранитель может попросить совершить ``противостоящий бросок''.

Для того, чтобы его совершить, обе стороны кидают бросок на навык и сравнивают уровни своего успеха. Таким образом, Обычный успех бьет Неудачу, Трудный успех бьет Обычный, а Экстремальный бьет Трудный. В случае ничьей побеждает тот, чье значение навыка выше. Если ещё и значения навыка одинаковые --- то решает бросок 1D100. Чье значение ниже --- тот и выиграл.

\subsubsection*{Кубик Преимущества и Кубик Помехи}

Окружение, условия и/или доступное время могут как помочь, так и помешать броску. При определенных условиях Хранитель может предоставить ``преимущество'' либо ``помеху'' к броску. Каждая помеха вычитает одно преимущество. Преимущества и помехи действуют подобно увеличенной сложности и могут использоваться вместе с ней или вместо неё. Но, как правило, их используют для противостоящих бросков.
\smallbreak
\smallbreak
\noindent\textbf{За каждый кубик преимущества:} киньте дополнительный 1D10, отвечающий за десятки, а не единицы, вместе с кубиками, используемыми в броске. То есть вы должны кинуть три кубика: один с единицами и два с десятками. Чтобы использовать преимущество, выберите лучший показатель на кубиках с десятками (более низкий).
\smallbreak
\noindent \example{два соперничающих исследователя, Малькольм и Хью борются за внимание Леди Грин. Только один из них может рассчитывать на её благосклонность и руку, так что Хранитель определяет, что для понимания того, кто оказывается более успешным ухажером, требуется противостоящий бросок. Игроки и Хранитель договариваются, что наиболее подходящим будет бросок на Обаяние. Хранитель вспоминает всё, что произошло в ходе сценария к этому моменту: Малькольм посетил Леди Грин дважды, не забывая о дорогих подарках, в то время как Хью посетил её лишь единожды и не позаботился о внешнем виде. Хранитель определяет, что Малькольм имеет преимущество и может использовать его в броске.

Игрок, играющий за Хью, сперва кидает обычный бросок на навык. Значение его Обаяния равно 55, и игрок выбрасывет 40 на десятках и 5 на единицах, всего 45 --- Обычный успех.

Игрок Малькольма также бросает обычный бросок на навык, но с преимуществом. Его Обаяние равно 50. Он кидает три кубика: два на десятки и один на единицы. Кубик с единицами показывает значение 4, а на десятках выпало 20 и 40. Не долго думая, игрок выбирает 20, в сумме 24 --- Трудный успех.

Малькольм побеждает в противостоянии, и его сватовство к Леди Грин принято благосклонно.
}
\smallbreak
\smallbreak
\noindent\textbf{За каждый кубик помехи:} киньте дополнительный 1D10, отвечающий за десятки, со стандартной парой кубиков. То есть вы должны кинуть три кубика: один с единицами и два с десятками. Чтобы получить помеху, вы должны выбрать худший показатель на кубиках с десятками (более высокий).
\smallbreak
\noindent \example{из-за череды неудачных стечений обстоятельств, два исследователя, Феликс и Гаррисон, оказались в плену у членов культа Алой Улыбки. Культисты решили ``поразвлечься'' с исследователями и хотят, чтобы они прошли Ордалию Боли, в ходе которой выживет только один. Проигравший будет принесен в жертву омерзительному божеству культа.

Ордалия Боли заключается в том, что нужно как можно дольше продержать тяжёлый камень. Это требует противостоящего броска на СИЛ от обоих исследователей. Однако Хранитель определяет, что Гаррисон будет совершать бросок с помехой, ведь он недавно получил серьёзную травму (как раз тогда, когда был захвачен культистами) и всё ещё не восстановился.

Игрок Феликса выкидывает 51 на своем броске по СИЛ. Его показатель СИЛ равен 60 --- у него Обычный успех.

Сила Гаррисона --- 55. Его игрок выкидывает 20 и 40 на кубиках с десятками и 1 на кубике с единицами. Он должен использовать более высокий результат из-за помехи --- в итоге у Гаррисона тоже Обычный успех.

У обоих игроков Обычный успех, но Феликс выигрывает, т.к. его показатель характеристики СИЛ выше. Феликс удержал камень над головой дольше, чем изнемогающий от боли Гаррисон. Культисты визжат от удовольствия и тащат обессиленного Гаррисона к алтарю...
}

\subsubsection*{Броски Удачи}

Хранитель может просить совершить бросок на Удачу, когда нужно определить обстоятельства вне контроля исследователей и отыграть слепую руку судьбы. Если, например, исследователь хочет узнать, есть ли рядом предмет, который можно использовать в качестве оружия, или остался ли ещё заряд в только что найденном фонарике, тогда нужен бросок на Удачу. Обратите внимание, если в ситуации уместнее кинуть навык или характеристику, следует кидать на них, а не на Удачу. Чтобы бросок на Удачу считался успешным, нужно выкинуть больше либо равно текущему показателю Удачи исследователя.

Если Хранитель просит совершить ``групповой бросок на Удачу'', это означает, что исследователь с худшим (самым низким) показателем Удачи из участвующих в сцене должен бросить кости от лица всей группы.
\smallbreak
\noindent \example{найти кэб --- не требует броска кубиков, но найти кэб до того, как подозреваемые скроются на своей машине, --- это уже зависит от броска Удачи. Показатель навыка Благосостояния может сыграть свою роль в привлечении внимания таксиста, который ищет щедрого на чаевые клиента. Но возможность быстро найти такси в два часа утра в нехорошем квартале города скорее будет зависеть от Удачи. И вообще, найдется ли кэб? Нет навыка, который отвечал бы за местонахождения кэбов, --- на это целиком воля судьбы, так что здесь требуется бросок Удачи.}

\section*{Рассудок (РАС)}

Когда исследователь сталкивается с необъяснимыми ужасами Мифологии Ктулху или с чем-то естественным, но устрашающим (таким как вид обезображенного трупа друга), нужно прокинуть 1D100 по вашему текущему показателю рассудка. Если вы выбросите больше --- вы потеряете большее число очков Рассудка, если меньше --- либо вообще не потеряете, либо потеряете меньше. Потеря очков Рассудка обычно описывается в готовых сценариях как ``0/1D6'' или ``2/1D10''. Число слева от слеша говорит о том, сколько вы потеряете в случае успеха, а справа --- сколько потеряете в случае неудачи.

\begin{fullcocpaperbox}{}{}
  \raisebox{-10mm}
  
  \sepline
  \cocpaperboxtitle{Примеры фобий и маний}
  Существуют сотни возможных фобий и маний. Здесь приведены лишь некоторые из возможных.
  \begin{multicols}{2}
  \subsubsection*{Фобии}
  \begin{itemize}[leftmargin=4mm]
  \item Боязнь высоты (акрофобия).
  \item Боязнь пауков (арахнофобия).
  \item Боязнь книг (библиофобия).
  \item Боязнь зеркал (эйзоптрофобия).
  \item Боязнь крови (гемафобия).
  \item Боязнь мёртвых существ (некрофобия).
  \item Боязнь зубов (одонтофобия).
  \item Боязнь огня (пирофобия).
  \item Боязнь телефонов (телефонофобия).
  \item Боязнь незнакомцев или чужаков (ксенофобия).
\end{itemize}
  \bigbreak
  \bigbreak
  \bigbreak  \bigbreak
  \bigbreak
  \bigbreak
  \subsubsection*{Мании}
  \begin{itemize}[leftmargin=4mm]
  \item Патологическая вежливость (агатомания).
  \item Одержимость болью (алгомания).
  \item Иррациональная жизнерадостность (аменомания).
  \item Одержимость кражей книг (библиоклептомания).
  \item Одержимость вершением справедливости (дикемания).
  \item Неконтролируемое желание смеяться (гелиомания).
  \item Иррациональное желание кричать (клазомания).
  \item Иррациональная потребность в кражах (клептомания).
  \item Уверенность в наличии воображаемой болезни (носомания).
  \item Иррациональное желание лгать (псевдомания).
\end{itemize}
  \end{multicols}
  \sepline
\end{fullcocpaperbox}

Когда вы проваливаете бросок на Рассудок, Хранитель отыгрывает ваше следующее действие самостоятельно, ведь вас охватывает неконтролируемый страх: возможно, вы неожиданно вскрикиваете или невольно нажимаете на курок вашего револьвера.

Если исследователь теряет 5 или больше очков Рассудка в ходе одного броска на Рассудок, он получает серьезную эмоциональную травму. Игрок должен кинуть 1D100. Если результат меньше или равен показателю Интеллекта (ИНТ) его персонажа, это означает, что персонаж вполне понял, что перед ним предстало, и впадает во временное безумие (длительностью в  1D10 часов). Если персонаж провалил бросок, его разум инстинктивно отказывается воспринимать ужас и он остаётся (пока что) в здравом уме.

Вдобавок, безумный исследователь страдает от ``приступа безумия'' --- киньте 1D10 и обратитесь к Таблице Приступов Безумия (стр. 15). Если исследователь находится в компании своих товарищей, отыгрывайте эффект от раунда к раунду. Если исследователь один, можете использовать этот результат, чтобы описать, как вашего персонажа находят через некоторое время в плохом состоянии, возможно, запершимся изнутри в шкафу или пьяным вдрызг в канаве.
% \begin{strip}
\begin{fullcocpaperbox}{}{}
  \raisebox{-10mm}
  
  \sepline
  \cocpaperboxtitle{Таблица Приступов Безумия}
  Хранитель выбирает самостоятельно или путем броска 1D10.
  \begin{multicols}{2}
  \begin{enumerate}
    \item \textbf{Амнезия:} исследователь не помнит о событиях, которые произошли с момента его последнего пребывания в безопасном месте.
    \item \textbf{Психосоматическая инвалидность:} исследователь испытывает психосоматическую слепоту, глухоту или неспособность управлять конечностью на 1D10 раундов.
    \item \textbf{Беспричинная жестокость:} кровавая пелена застилает глаза исследователя, и под воздействием неконтролируемой ярости персонаж начинает набрасываться на всё, что оказывается под рукой: и на союзников, и на врагов. Эффект длится 1D10 раундов.
    \item \textbf{Паранойя:} Исследователь испытывает сильную паранойю на протяжении 1D10 раундов. Все хотят ему навредить! Никому нельзя доверять! За ним следят, его предали, все, что он видит, --- это обман.
    \item \textbf{Важное лицо:} посмотрите в список Важных Лиц на листе персонажа. Исследователь принимает другого человека, присутствующего в сцене, за одно из этих Важных Лиц. Исследователь действует в соответствии со своим отношением к нему. Эффект длится 1D10 раундов.
    \item \textbf{Обморок:} персонаж теряет сознание, приходя в себя через 1D10 раундов.
    \item \textbf{Паническое бегство:} исследователя охватывает неодолимое желание бежать куда глаза глядят, как можно дальше от этого места, даже если ему придется использовать единственное доступное транспортное средство, оставив товарищей на произвол судьбы.
    \item \textbf{Истерика:} исследователь охвачен смехом, его душат рыдания, он кричит во всю мощь легких и т.п. на протяжении 1D10 раундов.
    \item \textbf{Фобия:} персонаж приобретает новую фобию, например, клаустрофобию (боязнь замкнутых пространств), демонофобию (боязнь духов / демонов) или катсаридофобию (боязнь тараканов). Даже если источника фобии нет рядом, исследователь думает, что это не так на протяжении 1D10 раундов, и все действия совершаются с помехой на протяжении действия эффекта.
    \item \textbf{Мания:} исследователь приобретает новую манию, например, аблютоманию (неодолимое желание вымыться), псевдоманию (тягу ко лжи) или гельминтоманию (одержимость любовью к червям). Исследователь стремится, удовлетворить манию на протяжении ближайших 1D10 раундов, и все действия приобретают помеху, пока длится эффект.
  \end{enumerate}
  \end{multicols}
  \sepline
  \end{fullcocpaperbox}{}{}
% \end{strip}
Если ваш исследователь временно безумен, Хранитель может решить добавить фобию или манию на вашем листе персонажа (например, ``боязнь темноты'', ``боязнь замкнутых пространств'' или ``клептоманию''). Как вариант, Хранитель может переписать одну из граф в вашем жизнеописании, искажая её (там, где вы написали ``доверчивый'' в качестве Черты, Хранитель может переписать на ``боязливый'').

Пока исследователь находится в состоянии временного безумия, Хранитель может наградить его ``Бредом'' (галлюцинациями), например, персонаж не может понять --- за его ногу цепляется зомби или это лишь бездомный человек попросил мелочи? В таком случае игрок может быть уверен, только если он прокинет ``определение действительности'', для чего нужно опять совершить бросок на Рассудок. Если он успешен, персонаж может понять, что реально, а что нет. Если не успешен, --- персонаж погружается ещё глубже в пучину безумия!

После прохождения 1D10 часов, исследователь приходит в себя и не может быть подвержен галлюцинациям, но измененное жизнеописание, фобии и мании остаются.
  \begin{fullcocpaperbox}{}{}

  \sepline
  \cocpaperboxtitle{Памятка Рассудка}
  \begin{multicols}{2}
    Когда совершается бросок на Рассудок, потеря отображается в таком виде: XX/XX (например, 1/1D6). Число слева --- это количество очков Рассудка, которое будет потеряно в случае успешного броска, справа --- в случае неудачи.

    \textbf{Непроизвольное действие:} любая потеря очков Рассудка приведёт к непроизвольному действию, определяемому Хранителем (невольный вскрик, обморок на один раунд, случайное нажатие на спусковой крючок).

    \textbf{Если в ходе броска было потеряно больше 5 очков Рассудка:} попросите исследователя совершить бросок на Интеллект (ИНТ): если его постиг провал, персонаж остается в здравом рассудке; если бросок был успешен (т.е. игрок выкинул значение, равное или меньшее своему показателю ИНТ),  исследователь временно теряет рассудок (на 1D10 раундов). Хранитель может применить нижеследующее, как ему самому больше нравится:

    \begin{enumerate}
    \item Приступ безумия: выберите или бросьте на один из вариантов в Таблице Приступов Безумия и примените результат на 1D10 раундов.
    \item (Опционально): возьмите лист персонажа и добавьте ему подходящую случаю запись в жизнеописание или измените существующую, основываясь на виде безумия или его причине.
    \item  (Опционально): добавьте к жизнеописанию персонажа фобию или манию.
    \item  Опишите галлюцинации, которые может испытывать персонаж в случае, если он не сможет прокинуть ещё один бросок на Рассудок. Галлюцинации могут преследовать персонажа не более чем 1D10 часов.
    \end{enumerate}
  \end{multicols}
  \sepline
\end{fullcocpaperbox}

\section*{Бой}

Когда вам противостоят ужасы Мифологии Ктулху, чаще всего верным решением будет развернуться и убежать. Ещё лучше --- изначально избегать конфронтации, потому что большинство чудовищ крайне могущественны и обычной пулей им не навредить. Тем не менее, иногда нет иного выхода, кроме как заявиться к культистам с оружием наперевес.

Во время боя все персонажи --- как исследователи, так и их оппоненты --- ходят в порядке показателя ЛВК. Чем выше этот показатель, тем раньше действует персонаж.

\smallbreak
\noindent \example{Персонажу Лёши противостоит культист, только что призвавший чудовище. Кажется, назревают неприятности. У исследователя показатель ЛВК равен 50, у культиста --- 45, у чудовища --- 70. Таким образом, чудовище в рамках раунда боя будет ходить первым, персонаж Алексея --- вторым, культист --- третьим.}

Продолжительность раунда боя в {\it Зове Ктулху} может быть описана как ``достаточная, чтобы все совершили одно действие''. Хранитель контролирует течение боя. В ход каждого персонажа Хранитель решает (или спрашивает у игрока, если ходит не NPC), что он будет делать. Чаще всего это нечто простое, вроде ``я нападаю на чудовище'', ``я достаю револьвер'' или ``я убегаю''. Хранитель должен предоставить всем возможность сделать что-то быстрое, стараясь не замедлять ритм повествования.

У исследователей есть три боевых навыка: "Бой", "Уклонение" и "Огнестрельное Оружие". Два из этих навыков состоят из нескольких специализаций, например, Бой (Рукопашный) или Огнестрельное Оружие (Винтовка / Дробовик). Каждый игрок в момент распределения очков по навыкам должен решить, имеет ли его персонаж специализацию и, если имеет, то какую. Обратите внимание, что под специализацию Бой (Рукопашный) подпадает бой без оружия и использование простого вооружения вроде ножей и дубин, однако более специализированное оружие наподобие меча потребует специализации Бой (Меч).

Вы совершаете броски в бою, используя наиболее подходящие навыки, как и в случае с любыми другими бросками. Единственное исключение: их нельзя пушить.

\subsection*{Ближний Бой}

В ход персонажа в порядке показателя ЛВК он может инициировать атаку против оппонента. Кроме того, каждый раз, когда на персонажа нападают, он может выбрать свою реакцию: уклонение (пытаясь полностью избежать урона), контратака (стараясь избежать, блокировать или парировать атаку, в то же самое время стремясь дать сдачи).

И нападающий, и защищающийся кидают процентный кубик (1D100) и сравнивают результаты бросков:

\begin{itemize}[leftmargin=4mm]
  \item Если вы контратакуете, вы используете свой навык Боя. Вам нужно достичь большего уровня успеха, чем у оппонента.
  \item Если вы уклоняетесь, вы используете навык Уклонения. Атакующий вас персонаж должен достичь большего уровня успеха, чем вы.
\end{itemize}

Всё довольно просто: выигравшая сторона не получит никакого урона и нанесет урон (если она не уклонялась) своему противнику.

Когда персонаж контратакует, самый большой уровень урона, который он может нанести, --- это ``обычный'', в то время как персонаж, инициирующий атаку, может достичь максимум ``экстремального'' урона (см. ниже).

\smallbreak
\noindent \example{Вурдалак пытается попасть своей когтистой лапой по персонажу Полины, Полина хочет уклониться. Хранитель выкидывает 03 --- Экстремальный успех (значение, меньшее пятой показателя Боя вурдалака). Полина выкидывает 20 на своём броске Уклонения --- Трудный успех. Нападающий достиг более высокого уровня успеха, чем уклоняющийся, поэтому по исследователю Полины попали, и он получает максимум урона --- 10 (1D6 + 1D4), потому что вурдалак достиг Экстремального успеха.

Вурдалак --- чудовище, которое может выполнить 3 атаки за раунд (все они происходят одновременно). Своей следующей атакой он пытается укусить персонажа Полины, которая хочет контратаковать. Полина достигает Трудного успеха, а вурдалак --- Обычного. Полина преуспела больше, чем её оппонент, поэтому она не только избегает новых повреждений, но и наносит 1D3 урона врагу.

Помните: если результат броска на Уклонение равен результату броска на атаку нападающего, выигрывает уклоняющийся. Если результат броска на контратаку равен результату броска нападающего, то побеждает атакующий, а не контратакующий. Уклоняться проще, чем контратаковать.
}
\smallbreak

\subsubsection*{Экстремальный Урон}

Атаки, которые достигли Экстремального уровня успеха, наносят увеличенный урон.

\begin{itemize}[leftmargin=4mm]
  \item Тупое оружие наносит максимальный урон плюс максимальный бонус к урону (если он есть).
  \item Проникающее оружие (клинки и пули) наносит максимальный урон плюс максимальный бонус к урону (если он есть) плюс дополнительный бросок на урон оружием (например, для пистолета --- 1D10 + 10 урона).
\end{itemize}

\smallbreak
\noindent \example{Персонаж Ромы выигрывает раунд боя с Экстремальным успехом. У него в руках дубина (тупое оружие), и его бонус к урону равен 1D4. Атака наносит 6 + 4 = 10 урона. Если бы он использовал нож (проникающее оружие), то нанес бы 4 + 4 + 1D4 урона.
}
\smallbreak

\subsection*{Правила для огнестрельного оружия}

Стреляющий из огнестрельного оружия кидает процентный кубик и сравнивает результат со своим показателем навыка Огнестрельное Оружие.

\begin{itemize}[leftmargin=4mm]
  \item Если оружие было взято наизготовку, то при определении очередности хода это дает бонус в +50 к показателю ЛВК.
  \item Если вы совершаете в один раунд 2 или 3 выстрела из пистолета, вы должны добавить один кубик помехи к каждому выстрелу.
  \item При выстреле в упор (выстрел считается таковым, если расстояние до цели меньше 1/5 ЛВК в футах), стреляющий получает кубик преимущества на бросок.
\end{itemize}

Цель выстрела не может выбрать реакцию (невозможно увернуться от пули), но можно попытаться кинуться в укрытие, сделав бросок на Уклонение. Если бросок успешен, то атакующий совершает броски на попадание с кубиком помехи. Персонаж, который решает спрятаться, теряет возможность совершить свою следующую атаку (независимо от того, успешно ли он укрылся). Если персонаж уже совершил свою атаку в этом раунде, то он теряет возможность совершить атаку в следующем.

\smallbreak
\noindent \example{Исследователь Лёши взял револьвер наизготовку, а затем увидел, что к нему, размахивая ятаганом, несется культист. ЛВК персонажа равна 50, но его взведённое оружие дает бонус в 50, давая показатель в 100 при определении очерёдности хода. ЛВК культиста всего 45, так что исследователь Алексея будет ходить первым.

Культист видит револьвер и бросается в укрытие (совершая бросок на Уклонение), что ему успешно удаётся. Лёша совершает свой бросок с помехой, в результате проваливая его. Культист потерял своё действие в раунде, т.к. бросился в укрытие, поэтому раунд заканчивается. Начинается новый раунд, что даёт Алексею ещё одну возможность прострелить культисту голову.
}

\begin{cocpaperbox}{}{}
  \raisebox{-10mm}
  
  \sepline
  \cocpaperboxtitle{Оружие и урон}
  \bigbreak
\begin{itemize}[leftmargin=4mm]
  \item Безоружные атаки (человек): 1D3 + бонус к урону
  \item Небольшой нож: 1D4 + бонус к урону
  \item Мачете: 1D8 + бонус к урону
  \item Небольшая дубина: 1D6 + бонус к урону
  \item Бейсбольная бита: 1D8 + бонус к урону
  \item Пистолет: 1D10
  \item Дробовик: 4D6 (на короткой дистанции*, в противном случае 2D6, без проникновения)
  \item Винтовка: 2D6 + 4
\end{itemize}
\bigbreak
*Короткая дистанция: в пределах показателя ЛВК в футах (т.е. если ЛВК равна 60, то короткой дистанцией считается расстояние в 60 футов (18 метров)).
  \sepline
\end{cocpaperbox}

\smallbreak

\subsection*{Маневры в бою}

Если игрок заявляет цель, отличную от нанесения обычного физического урона, подобная цель может быть описана путем ``Боевого Манёвра''. Успешный манёвр позволяет игроку достичь такой цели как, например:

\begin{itemize}[leftmargin=4mm]
  \item Разоружить противника,
  \item Сбить противника с ног,
  \item Схватить и удерживать врага, из-за чего противник должен совершать все броски с помехой, пока не избавится от захвата.
\end{itemize}

Манёвр считался стандартной атакой, бросок совершается по навыку Бой (Рукопашный). Оппонент, как и обычно, может выбрать уклонение или контратаку. Затем сравнивается Комплекция. Если персонаж, который совершает манёвр, имеет меньший показатель Комплекции, то он совершает манёвр с дополнительным кубиком помехи за каждое очко разницы (до максимума в два кубика помехи). Если персонаж имеет показатель Комплекции, меньший Комплекции противника на 3 или больше, то манёвр бесполезен: персонаж может схватить оппонента, но у него не хватит сил, чтобы сдвинуть его с места.

\smallbreak
\noindent \example{Полина пытается вытолкнуть вурдалака из ближайшего окна (рассматривается как манёвр). Комплекция персонажа Полины равна 0, Комплекция вурдалака 1, поэтому Полина совершает бросок с помехой. Полина выкидывает 02 и 22, из-за помехи она вынуждена выбрать худшее (наивысшее) значение, что приносит ей Трудный успех (22 меньше 1/2 показателя навыка Боя Полины). Вурдалак совершает контратаку и выбрасывает Обычный успех на навыке Боя. Полина достигла более высокого уровня успеха, поэтому вурдалак с жутким воплем летит к земле.
}

\subsubsection*{Окружён}

Персонаж, которого окружили враги, оказывается в трудной ситуации. После того, как персонаж совершил уклонение или контратаку в текущем раунде, все последующие атаки по нему совершаются с преимуществом. Это правило не применяется к огнестрельному оружию.

\smallbreak
\noindent \example{Вурдалак имеет 3 атаки, исследователь Полины --- только 1. После первой атаки вурдалака Полина может выбрать, уклоняться или контратаковать. Но во время второй и третьей атаки вурдалака, он получает преимущество, т.е. по сути персонаж Полины расценивается как окруженный.

В другой ситуации Рома противостоит двум культистам, он тоже окружен. Атака первого культиста будет совершаться как обычно, а второй культист получит преимущество.
}

\section*{Очки здоровья, раны и исцеление}

Очки урона вычитаются из очков здоровья персонажа. Когда очки здоровья падают до нуля, исследователь теряет сознание и в некоторых случаях может умереть.

Если персонаж получает за один удар урон, равный или больший половине его полного значения очков здоровья, он получает Значительную Рану'' --- ему нужно прокинуть его значение ТЛС, иначе он потеряет сознание. Если исследователь со значительной раной достигает нуля очков здоровья, он оказывается при смерти. В конце каждого последующего раунда, включая текущий, ему придется бросать бросок на ТЛС, чтобы не умереть. Только успешное применение навыка Первой Помощи может снять это состояние, стабилизируя персонажа. Если персонаж получает за один удар урон, больший или равный его полному показателю очков здоровья, --- он умирает моментально.

Обратите внимание, что если у персонажа нуль очков здоровья, но нет Значительной Раны, то он не умирает --- смерть возможна только при наличии Значительной Раны.

\begin{itemize}[leftmargin=4mm]
  \item Персонажи без Значительной Раны исцеляют 1 очко урона в день естественным путем.
  \item Персонажи со Значительной Раной должны совершать бросок на исцеление (выбросив значение, меньшее или равное их показателю ТЛС) в конце каждой недели. При успехе они восстанавливают 1D3 очков здоровья (2D3 в случае Экстремального успеха). Значительная Рана исчезнет, если персонаж выбросит на броске исцеления Экстремальный успех или текущее значение очков здоровья достигнет половины или более их максимального значения. Таким образом, для полного исцеления могут потребоваться недели.
\end{itemize}

Успешное применение навыка Первой Помощи также может восстановить 1 очко здоровья или привести персонажа в чувство. Если персонаж был при смерти, Первая Помощь может его стабилизировать, а затем при помощи навыка Медицины персонажу можно восстановить 1D3 очков здоровья, но для этого также потребуется минимум один час и необходимые инструменты и лекарства. Если навык Медицины используется на умирающем персонаже, в конце недели он сможет совершить бросок на исцеление.

\smallbreak
\noindent \example{Изначально у персонажа Лёши 12 очков здоровья. В понедельник в баре он оказывается участником драки и получает урон от трех ударов в челюсть, стоящих ему 4, 2 и 4 очка здоровья. Всего это 10 очков здоровья, у исследователя Алексея их остается 2. Он не получил Значительной Раны (ни одна атака не нанесла большого количества урона), так что он будет восстанавливаться со скоростью 1 очко здоровья в день. В четверг персонаж (теперь с 5 очками здоровья) по неловкости выпадает из окна, получая 7 очков урона. Это --- Значительная Рана (7 больше половины максимального количества очков здоровья персонажа). Друг применяет Первую Помощь и спешно отвозит его в больницу. По прошествии семи дней Алексей успешно прокидывает ТЛС и на броске 1D3 выкидывает 2, чем восстанавливает 2 очка здоровья. В конце второй недели Алексей выбрасывает Экстремальный успех на броске ТЛС и получает 4 очка здоровья на броске 2D3, его текущее количество здоровья становится равным 6. Это убирает у него Значительную Рану (он восстановил половину своего максимального значения очков здоровья), после чего он исцеляется со скоростью 1 очко здоровья в день.
}

\subsection*{Прочие виды урона}

Часто Хранителю приходится определять количество урона, которое персонажи получают от разнообразных случайных происшествий. Независимо от причины, определитесь с возможной травмой и оцените её по левой колонке в Таблице Прочих Видов Урона. Каждая травма связана с конкретным происшествием или определенным раундом боя (например, персонаж может иметь одну травму от раунда, в котором в него попала пуля, ещё одну --- от другого раунда, в котором его топили, третью --- от раунда, в котором он горел, и т.д.).

\begin{fullcocpaperbox}{}{}
  \raisebox{-10mm}
  
  \sepline
  \cocpaperboxtitle{Таблица Прочих Видов Урона}
  \rowcolors{2}{}{TableRow}
  \begin{tabularx}{170mm}{XYX}
  \rowcolor{TableHeading}\color{CoCPaperBox}\bfseries Травма & \color{CoCPaperBox}\bfseries Урон & \color{CoCPaperBox}\bfseries Примеры \\
  \textbf{Малый:} человек может выжить, даже испытав урон подобного уровня несколько раз. & 1D3 & Удар рукой, ногой, головой, слабая кислота, нахождение в задымлённом помещении*, удар камнем средних размеров, падение (урон рассчитывается за каждые 10 футов (т.е. 0.3 метра) падения) в мягкую болотную почву. \\
  \textbf{Средний:} может привести к серьёзной травме, испытав урон подобного уровня несколько раз, человек может умереть. & 1D6 & Падение (урон рассчитывается за каждые 10 футов (т.е. 0.3 метра) падения) на траву, удар дубиной, сильная кислота, дыхание под водой*, нахождение в вакууме*, малокалиберная пуля, стрела, огонь (прикосновение к огню). \\
  \textbf{Тяжёлый:} скорее всего приведет к тяжелой травме. Один или два могут привести к смерти. & 1D10 & Пуля калибра .38, падение (урон рассчитывается за каждые 10 футов (т.е. 0.3 метра) падения) на бетон, удар топором, огонь (атака из огнемета, бег через горящую комнату), нахождение в 6-9 метрах от разорвавшейся гранаты или динамитной шашки, слабый яд**.\\
  \textbf{Смертельная:} Среднестатистический человек умрет в 50\% случаев. & 2D10 & Попадание под машину на скорости 50 км/ч, нахождение в 3-6 метрах от разорвавшейся гранаты или динамитной шашки, сильный яд**. \\
  \textbf{Терминальная:} Вероятна моментальная смерть. & +4D10 &  Попадание под машину на полной скорости, нахождение в 3 метрах от разорвавшейся гранаты или динамитной шашки, смертельный яд**. \\
  \textbf{В лепешку:} Почти наверняка моментальная смерть. & +8D10 & Попадание в аварию на полной скорости, попадание под поезд. \\
  \end{tabularx}
\smallbreak
*Утопление и удушение: каждый раунд необходимо совершать бросок на ТЛС --- если бросок провален, то каждый последующий раунд наносится урон вплоть до смерти или до того момента, когда жертва сможет снова дышать. Смерть наступает при достижении нуля очков здоровья (правило Значительной Раны игнорируется).

**Яды: Экстремальный успех при броске ТЛС ополовинивает получаемый от яда урон.

  \sepline
\end{fullcocpaperbox}

\section*{Награда за успех}

Несмотря на то, что многие из сценариев к {\it Зову Ктулху} могут быть сыграны за один вечер, некоторые из них подразумевают, что исследователи будут расти и развиваться. В конце сценария или сессии Хранитель должен попросить сделать ``бросок на увеличение навыков''. В этот момент каждый игрок кидает процентный кубик (1D100) за каждый навык, отмеченный галочкой (т.е. те, которые на протяжении сессии использовались успешно). Если этот бросок будет провален (т.е. значение выше текущего показателя навыка), это означает, что исследователь получил новый опыт в его использовании и вы можете добавить 1D10 очков к этому навыку. Другими словами --- чем лучше вы владеете навыком, тем сложнее вам достичь в нём новых высот.

\smallbreak
\noindent \example{Анна (исследователь Маши) совершает успешный бросок на Наблюдательность в течение игры, и Маша помечает этот навык галочкой на листе персонажа. После окончания сценария Хранитель просит Машу кинуть кубик на увеличение навыка. Уровень Наблюдательности Анны равен 45\%, а бросок равен 43. Увеличения навыка не происходит. Если бы Маша выкинула больше 45, она бы могла добавить 1D10 очков к своему показателю Наблюдательности. Маша стирает галочку с листа персонажа. В случае успеха в проверке Наблюдательности в следующей игре она сможет отметить его снова.
}

\begin{fullcocpaperbox}{}{}
  \raisebox{-10mm}
  
  \sepline
  \cocpaperboxtitle{Описание навыков}
  \begin{multicols}{3}
  Большинство навыков означают ровно то, что написано в названии: например, Вождение Автомобиля определяет, насколько хорошо персонаж управляется с автомобилем, а Скалолазание --- насколько хорошо исследователь сможет вскарабкаться по стене здания. Тем не менее, наименование некоторых навыков может ввести в заблуждение, поэтому здесь приведено краткое описание навыков, указанных на листе персонажа.
  \smallbreak
  \textbf{Обратите внимание:} на листе исследователя рядом с каждым навыком в скобках приведено процентное значение: это показатель навыка по умолчанию. То есть любой человек может использовать пистолет (базовое значение Огнестрельного Оружия равно 20\%), даже если он раньше никогда не стрелял.
  \smallbreak
  \textbf{Археология:} умение датировать и опознавать артефакты древних культур, а также определять их подделки.
  \smallbreak
  \textbf{Антропология:} понимание образа жизни человека (или группы) путем наблюдения.
  \smallbreak
  \textbf{Благосостояние:} уровень Благосостояния персонажа показывает, насколько он богат и к какому слою общества он принадлежит. Чем больше очков потрачено на этот навык, тем он богаче. В зависимости от того, сколько очков профессиональных навыков вы потратите, ваш персонаж будет:
    \begin{itemize}[leftmargin=4mm]
      \item Благосостояние 0: без гроша в кармане, живет на улице.
      \item Благосостояние 1-9: бедняк, владеет самым минимумом.
      \item Благосостояние 10-49: среднего достатка, может обеспечить себе сносный уровень жизненного комфорта.
      \item Благосостояние 50-89: зажиточный, может позволить себе некоторую степень роскоши.
      \item Благосостояние 90-98: богатый, огромные доходы и непозволительная для всех остальных роскошь.
      \item Благосостояние 99: сказочно богатый, деньги не имеют значения.
    \end{itemize}
  \smallbreak
    \noindent \example{Рома выбирает своему солдату уровень Благосостояния равный 40\% --- это означает, что у того будет средний уровень дохода.
    }
  \smallbreak
  \textbf{Бой:} умение персонажа драться в ближнем бою. Вы можете выбрать любую специализацию, такую как Рукопашный бой (сюда входят ножи и дубины, кулачный бой и боевые искусства), бой на Мечах, Топорах, Копьях или Кнутах.
  \smallbreak
  \textbf{Бросок:} умение попадать по цели брошенным предметом. Предмет размером с ладонь может быть брошен на расстояние в метрах, равное СИЛ персонажа, поделенной на 5.
  \smallbreak
  \textbf{Верховая Езда:} навык применяется к поездкам на взнузданных лошадях, ослах и мулах. Он гарантирует знание основ ухода за животным, снаряжением, умение управлять животным в галопе или на пересеченной местности. Если животное внезапно остановится или взбрыкнет, этот навык позволит наезднику остаться в седле.
  \smallbreak
  \textbf{Вождение Автомобиля:} умение водить легковой автомобиль или грузовик, совершать стандартные маневры и устранять обычные неисправности. Если исследователь хочет сбросить преследователя с хвоста или, наоборот, догнать чужую машину, бросок на Вождение будет вполне уместным.
  \end{multicols}
  \sepline
\end{fullcocpaperbox}

\begin{fullcocpaperbox}{}{}
  \raisebox{-10mm}
  
  \sepline
  \cocpaperboxtitle{Описание навыков}
  \begin{multicols}{3}
  \textbf{Выживание:} знания, необходимые для выживания в экстремальных условиях, таких как пустыня, море, пустошь или полярная тундра. В этот навык входит умение охотиться, строить укрытия, понимание естественных угроз (какие ягоды есть можно, а какие --- нет) и т.п. Очки навыков тратятся на определенные специализации, такие как Пустошь, Арктика, Пустыня, Море и т.д. Когда у персонажа нет конкретной специализации, он может совершать броски по наиболее близкой специализации с увеличенной сложностью (или с кубиком помехи), если того захочет Хранитель.
  \smallbreak
  \textbf{Выслеживание:} умение следовать за человеком, транспортом или животным по земле или сквозь заросли. Факторы, влияющие на сложность: время, прошедшее с момента прохода того, кого выслеживают, дождь, тип поверхности и т.д.
  \smallbreak
  \textbf{Искусство и Ремесло:} возможность создавать или ремонтировать предметы или сущности, которые могут быть произведением искусства (например, картина или песня) или ремесла (например, столярного или кулинарного). Выберите подходящую специализацию и запишите в месте, предоставленном на листе персонажа.
  \smallbreak
  \textbf{Использование Библиотек:} умение находить в библиотеках информацию, такую как определенная книга, газета, полезная ссылка (если в конкретной библиотеке подобная информация присутствует). Использование этого навыка подразумевает трату нескольких часов, потраченных на поиски.
  \smallbreak
  \textbf{История:} способнось вспомнить деталь или событие из истории страны, города, региона или личности, в зависимости от обстоятельств.
  \smallbreak
  \textbf{Ловкость Рук:} позволяет спрятать или замаскировать объект или объекты одеждой, лохмотьями и другими подходящими материалами. Также отвечает за аккуратность при управлении тонким оборудованием.
  \smallbreak
  \textbf{Маскировка:} умение выдавать себя за кого-то ещё.
  \end{multicols}
  \sepline
\end{fullcocpaperbox}

\begin{fullcocpaperbox}{}{}
  \raisebox{-10mm}
  
  \sepline
  \cocpaperboxtitle{Описание навыков}
  \begin{multicols}{3}
  \textbf{Медицина:} способность диагностировать и лечить последствия несчастных случаев, травм, болезней, отравлений и т.д. Лечение требует как минимум одного часа и может применяться в любое время после получения урона, но если навык не применен в тот же день, сложность броска увеличивается (см. стр. 7) и требует Трудного успеха. Персонаж, успешно полеченный Медициной, восстанавливает 1D3 очка здоровья (в дополнение к полученным от Первой Помощи ) --- кроме случаев, когда персонаж при смерти, тогда до применения Медицины исследователь должен быть стабилизирован успешным броском Первой Помощи.
  \smallbreak
  \textbf{Механика:} умение чинить или создавать механизмы. Персонаж может осуществлять базовые кровельные и сантехнические работы, создавать инженерные устройства, такие как блок или паровая помпа. Навык может использоваться для открытия стандартных домовых замков, но для более сложных замков может потребоваться Слесарничество.
  \smallbreak
  \textbf{Мифология Ктулху:} понимание нечеловеческой природы Мифологии Ктулху. Он не базируется на приумножении знаний, в отличие от научных навыков. Он скорее описывает открытие и подстраивание человеческого разума к постижению истин Мифологии. Таким образом, очки навыков для Мифологии приобретаются от личных встреч с Мифологией (чудовищами или омерзительными тайнами, описанными в редких книгах). Секреты Мифологии с трудом воспринимаются человеческим разумом и беспощадны к психике. Ни один исследователь не может вкладывать в этот навык очки (если это заранее не оговорено с Хранителем).
  \textbf{Наблюдательность:} навык поможет найти тайную дверь или отсек, заметить спрятавшегося воришку, увидеть неброскую улику, понять, что автомобиль был недавно перекрашен, определить, что рядом засада, заметить оттопырившийся карман и т.д. Это один из наиболее важных навыков в арсенале исследователя. Когда исследователь ищет прячущегося персонажа, сложность броска определяется показателем Скрытности оппонента.
  \textbf{Навигация:} способность выбирать правильный путь к какой-либо точке, будь то в незнакомом городе или в глуши. Навык дает возможность эффективно читать карты и определять расстояния и ландшафт.
  \smallbreak
  \textbf{Наука:} практические и теоретические познания в фундаментальных науках, полученные путем формального образования, хотя им также могут владеть самообразованные ученые-любители. Масштаб знаний определяется эпохой, в которой происходит игра. Очки навыков используются для получения специализации --- например, Астрономии, Биологии, Ботаники, Химии, Криптографии, Геологии, Фармакологии, Физики, Зоологии и т.п. Когда у персонажа нет конкретной специализации, он может совершать броски по наиболее близкой специализации с увеличенной сложностью (или с кубиком помехи), если этого захочет Хранитель.
  \smallbreak
  \textbf{Обаяние:} совокупность того, как персонаж выглядит, его умения соблазнять, льстить или просто производить приятное впечатление. Обаяние может быть использовано, чтобы заставить кого-то вести себя определенным образом, но при этом не в манере, полностью противоположной его обычному поведению. Этому навыку можно противостоять Обаянием или Психологией.
  \end{multicols}
  \sepline
\end{fullcocpaperbox}

\begin{fullcocpaperbox}{}{}
  \raisebox{-10mm}
  
  \sepline
  \cocpaperboxtitle{Описание навыков}
  \begin{multicols}{3}
  \textbf{Огнестрельное Оружие:} владение огнестрельным оружием, а также луками и арбалетами. Вы можете тратить очки навыков на любые специализации, включая Пистолеты, Винтовки / Дробовики, Луки или Арбалеты.
  \smallbreak
  \textbf{Оккультизм:} умение распознавать оккультную атрибутику, слова и концепции, народные традиции, а также определять древние тома с магией и оккультные коды. Навык помогает вспомнить тайные мистические знания, почерпнутые из старинных книг, учений или собственного опыта.
  \smallbreak
  \textbf{Оценка:} определение ценности конкретного предмета, включая то, насколько качественно он сделан, из каких материалов и в каком регионе.
  \smallbreak
  \textbf{Первая Помощь:} умение оказывать экстренную медицинскую помощь. Этот навык не может использоваться для лечения заболеваний (для этого требуется навык Медицина). Для того, чтобы это возымело эффект, Первая Помощь должна быть применена не позднее часа с момента получения ранения, в таком случае восстанавливается 1 очко здоровья, и персонаж может придти в сознание.
  \smallbreak
  \textbf{Пилотирование:} выберите специализацию, такую как Катер, Самолет или Дирижабль. Каждая специализация требует отдельной оплаты очками навыков. Навык отвечает за возможность безопасно управлять определенным видом транспорта.
  \smallbreak
  \textbf{Плавание:} умение оставаться на поверхности и плыть в воде или другой жидкости. Совершайте бросок по этому навыку только в случае экстренной  обстановки или опасности. Неудача на пушнутом броске на Плавание может привести к потере очков здоровья. Она так же может привести к тому, что персонажа утащит потоком или он вовсе утонет.
  \smallbreak
  \textbf{Право:} знания, относящиеся к законам, прецедентам, юридическим уловкам или судебным процедурам. Навык полезен при взаимодействии с полицией, юристами и судами.
  \smallbreak
  \textbf{Природа:} традиционное (ненаучное) знание и личные наблюдения фермеров, рыбаков, охотников и увлекающихся любителей. С помощью навыка можно идентифицировать растения, животных, в общих чертах их привычки и ареал обитания, определять их следы, запахи и звуки.
  \end{multicols}
  \sepline
\end{fullcocpaperbox}

\begin{fullcocpaperbox}{}{}
  \raisebox{-10mm}
  
  \sepline
  \cocpaperboxtitle{Описание навыков}
  \begin{multicols}{3}
  \textbf{Психоанализ:} владение разными видами психоэмоциональной терапии. Психоанализ может вернуть утраченные очки Рассудка: раз в игровой месяц, чтобы понять, насколько терапия помогла, подвергаемый ей персонаж должен кинуть 1D100 по значению навыка Психоанализа врача. Если бросок успешен, то пациент восстанавливает 1D3 очка Рассудка. Если неуспешен --- ничего не происходит. Если бросок критически провален, пациент теряет 1D6 очков Рассудка и лечение у текущего специалиста прекращается. В игре невозможно избавиться от неопределенного безумия (этот момент не описан в правилах Рассудка в стартере, к сожалению) --- лечение от него требует пребывания 1D6 месяцев в психоневрологическом учреждении, на протяжении этого периода психотерапия будет лишь частью общего курса лечения. Успешное применение этого навыка также может позволить персонажу на короткое время справиться с приступом фобии или мании или не попасть под власть иллюзий.
  \smallbreak
  \textbf{Психология:} эмпатия, которая есть у всех людей, позволяющая понять мотивы действий другого человека и его характер или осознать, когда он лжет. Хранитель может делать броски на Психологию в закрытую, сообщая правдивую или ложную информацию, как её воспринимают персонажи.
  \smallbreak
  \textbf{Скалолазание:} способность карабкаться не только по скалам, но и по  деревьям, стенам и другим вертикальным поверхностям как с использованием снаряжения, так и без него.
  \smallbreak
  \textbf{Скрытность:} используется для избежания обнаружения, тихого передвижения и ухода от нежелательных лиц без привлечения их внимания.
  \smallbreak
  \textbf{Слесарничество:} умение открывать автомобильные двери, замыкать зажигание, взламывать библиотечные окна, разгадывать головоломки на комбинаторику и обходить стандартные системы сигнализации. Персонаж, хорошо владеющий этим навыком, может изготавливать ключи (при наличии образца), отмычки и прочие инструменты.
  \smallbreak
  \textbf{Слух:} интерпретация и понимание звуков. Сюда относится подслушивание разговоров, бормотания за закрытыми дверями и шепотов в кафе.
  \smallbreak
  \textbf{Счетоводство:} владение бухгалтерскими процедурами, носитель навыка может охарактеризовать финансовое состояние человека или компании.
  \end{multicols}
  \sepline
\end{fullcocpaperbox}

\begin{fullcocpaperbox}{}{}
  \raisebox{-10mm}
  
  \sepline
  \cocpaperboxtitle{Описание навыков}
  \begin{multicols}{3}
  \textbf{Убеждение:} умение внушить человеку конкретную идею, концепцию или верование путем взвешенного аргументирования, дискутирования или обсуждения. Убеждение можно использовать, не говоря правду. Успешное применение Убеждения требует времени, как минимум полчаса. Если вы хотите убедить кого-то на скорую руку --- используйте Увещевание.
  \smallbreak
  \textbf{Увещевание:} этот навык целенаправленно ограничен вербальными уловками, обманом и пусканием пыли в глаза. Он подходит для того, чтобы ошарашить вышибалу на входе в ночной клуб, уговорить кого-либо не глядя подписать документ, заставить полисмена закрыть глаза на происшествие и так далее. Этому навыку можно противостоять с помощью Увещевания или Психологии.
  \smallbreak
  \textbf{Уклонение:} способность инстинктивно избегать ударов, брошенных снарядов и т.д. Персонаж может попробовать уклониться любое количество раз за бой (но только один раз за атаку). Если атаку можно заметить, то от нее можно попробовать уклониться (т.е. от пуль, двигающихся со скоростью, не видной глазу, уклоняться нельзя; лучшее, что может сделать персонаж, --- это попытаться броситься в укрытие). Определите стартовое значение, поделив пополам ваше Обычное значение ЛВК.
  \smallbreak
  \textbf{Управление Тяжелой Машинерией:} требуется для управления поездами, паровыми двигателями, бульдозерами или другими большими машинами.
  \smallbreak
  \textbf{Электрика:} умение чинить и настраивать электрическое оборудование (например, автомобильное зажигание, электрические моторы, детонаторы, сигнализации).
  \smallbreak
  \textbf{Язык, Иной:} способность исследователя понимать, говорить, читать и писать на определенном языке, отличным от родного. При выборе этого навыка, нужно указать конкретный язык и отметить его рядом с навыком. Персонаж может знать любое количество языков, но за каждый необходимо отдельно платить очками навыков.
  \smallbreak
  \textbf{Язык, Родной:} вы выбираете любой язык, которым ваш персонаж владеет лучше всего. Стартовое значение равно обычному показателю ОБР.
  \end{multicols}
  \sepline
\end{fullcocpaperbox}

\end{document}